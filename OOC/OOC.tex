\chapter{Experiment 1: Bilinguals out of sentence context}\label{ESOOC}
The goal of this first experiment is to replicate previous studies showing that bilinguals access both languages nonselectively when words are presented in isolation.  A successful replication will ensure that the stimuli are capable of eliciting parallel activation, so that they can later be used to investigate bilingual word recognition in sentence contexts (Chapter \ref{SEInContext}).

Previous studies find evidence for nonselectivity by showing that  bilinguals, but not monolinguals, recognize cognates faster than matched control words. Cognate facilitation occurs as a result of the form and semantic overlap across the two languages. The logic of this experiment follows that of previous studies. Cognates and matched noncognate controls were selected, and bilinguals are asked to recognize and name the words presented isolation. Based on previous results, cognates in this experiment were predicted to be named faster than control words.

\section{Participants}
Sixteen participants from the Pennsylvania State University participated in this experiment for course credit. One participant was removed because their native language was not English. The remaining 15 participants were all native English speakers who were proficient L2 speakers of Spanish.

\section{Materials}
A set of 128 Spanish words were chosen for this experiment. These words consisted of 64 cognates (e.g., \textit{cable}) and 64 noncognate controls (e.g., \textit{chispa} -- \textit{spark}). The cognates and controls were matched for lexical properties such as length and frequency. The cognate words varied in the degree of orthographic overlap such that some cognates were identical (e.g., \textit{cable}) and others were nonidentical (e.g., \textit{catedral} -- \textit{cathedral}). The full list of materials is listed in Appendix \ref{Appendix::OOC}.

\section{Procedure}
Upon arrival to the lab, participants were asked to read and complete an informed consent form. After signing the informed consent, participants were asked to fill out a language history questionnaire. Participants were then seated at a computer and began the set of experiments, starting with the out of context word naming task. After the naming task, they completed a set of tasks designed to measure individual differences (working memory and executive control) and proficiency. Following participation, participants were given \$10 as compensation for their time. 

In the word naming task, participants were instructed both verbally and through written instructions on how to proceed through the task. A fixation cross (``+'') appeared before each word. Participants were told to press a button at each cross to bring up a Spanish word. They were told to name the word as quickly and accurately as possible as soon as it appeared. A voice-key trigger recorded the  latency to begin naming, and the entire session was recorded so naming accuracy could be computed later.  Ten practice trials preceded the experimental session to familiarize the participant with the task. During this time, the experimenter was present to answer any questions the participant might have. Following the practice section, the experimenter left the room. 

Following the word naming task, participants completed the Operation Span task, the Simon task, and the picture naming task outlined above in Chapter \ref{Roadmap}. Participant were then thanked for their participation and paid for their tuime. The experimental session lasted approximately 45 minutes.

\section{Results}
Data from the word naming task were analyzed by comparing naming latencies for cognates to naming latencies for matched controls. An analysis for accuracy was conducted in the same manner. Before these results are reviewed, data about the language experience and individual differences (working memory and executive function) are reviewed for the participants.

\subsection{Language history and individual difference data}
Data from the language history questionnaire, English picture naming task, Operation Span task, and Simon task  are shown in Table \ref{es_lhq_id}. The participants' ages ranged from 18 to 29 years with a mean age of 21.3. On average, they  began studying Spanish at 9.1 years of age  and had been studying for 11.8 years. 

% latex.table(x = as.matrix(lhq), file = "lhq") 
%
\begin{table}[hptb]
\begin{center}
\begin{tabular}{|l|l|} \hline
\multicolumn{1}{|l|}{English-Spanish Bilinguals}&\multicolumn{1}{l|}{}\\ \hline
N~~~~~~~~~~~~~~~~~~~~~~~~~~~&~15~~~~~~~~~~~~\\ 
Age~(years)~~~~~~~~~~~~~~~~~&~21.1~(2.6)~~~~\\ 
L2~Age~of~Onset~(years)~~~~~&~9.1~(6.1)~~~~~\\ 
L2~Length~of~Study~(years)~~&~11.8~(5.3)~~~~\\ 
Simon~Score~~~~~~~~~~~~~~~~~&~43.4~(26.0)~~~\\ 
Operation~Span~(Out~of~60)~~&~48.1~(7.1)~~~~\\ 
Picture~Naming~Accuracy~~~~~&~94\%~(0.06\%)~\\ 
\hline
\end{tabular}
\caption{Language background data and individual difference measures for English-Spanish bilinguals out of context}\label{es_lhq_id}
\end{center}

 






\end{table}

% latex.table(x = as.matrix(lr), file = "lr") 
%
\begin{table}[hptb]
\begin{center}
\begin{tabular}{|c|c|c|} \hline
\multicolumn{1}{|c|}{Language Ratings (Out of 10)}&\multicolumn{1}{c|}{Spanish (L2)}&\multicolumn{1}{c|}{English (L1)}\\ \hline
Reading~~~~&~7.5~(1.1)~&~9.7~(0.5)~\\ 
Spelling~~~&~7.8~(1.4)~&~9.6~(0.5)~\\ 
Writing~~~~&~7.3~(1.3)~&~9.6~(0.5)~\\ 
Speaking~~~&~7.4~(1.8)~&~9.8~(0.4)~\\ 
Listening~~&~7.9~(1.4)~&~9.9~(0.25)\\ 
Average~~~~&~7.7~(0.9)~&~9.7~(0.3)~\\ 
\hline
\end{tabular}
\caption{Self-assessed language ratings for English-Spanish bilinguals naming words out of context}\label{es_lr}
\end{center}
\end{table}


Self-reported language-proficiency ratings for reading, spelling, writing, speaking, and listening in each language are shown in Table \ref{es_lr}. These ratings were averaged over together for each participant to give an overall language rating. A paired t-test revealed that this group of participants judged themselves as more proficient in English (\textit{M} $=$ 9.7) than in Spanish (\textit{M} $=$ 7.7; \textit{t}(14) $=$ 9.68, \textit{p} $<$ 0.01). Participants scored very highly (\textit{M} $=$ 94\%) on the English picture naming task.  

\subsection{Word naming}
Before statistical analyses were conducted on the word naming data, latencies were trimmed of both absolute and relative outliers. Based on data scrubbing techniques of previous word naming studies, absolute cutoffs were set at 200 ms and 2000 ms. Trials outside of this range were removed from further analyses. Mean reaction times were calculated by participant by condition for correct trials. Trials with reaction times 2.5 \textit{SD} above or below this mean were marked as relative outliers and were removed from subsequent analyses. The outlier removal procedure resulted in the removal of 7\% of the data. Means and accuracies were then calculated by participant for cognate and control words. 

A paired-samples t-test revealed that latencies for cognate words (\textit{M} $=$ 615.8) were significantly faster than latencies for noncognate words (\textit{M} $=$ 647.3; \textit{t}(14) $=$ 4.14, \textit{p} $<$ 0.01). In terms of accuracy, there was no evidence that cognate and noncognate accuracies differed (cognates: 97\% correct, noncognates: 96\% correct; \textit{t}(14) $<$ 1, \textit{p} $>$ 0.5). The data from the word-naming task are reported in Table \ref{OOC_latencies}.

\begin{table}[hptb]
\begin{center}
\begin{tabular}{|c|c|c|} \hline
\multicolumn{1}{|c|}{}&\multicolumn{1}{c|}{Latency (in ms)}&\multicolumn{1}{c|}{Accuracy (\% correct)}\\ \hline
Cognates~~~&615.8~~~&0.97~~\\ 
Noncognates&647.3~~~&0.96~~\\ 
\hline
\end{tabular}
\caption{Naming latencies and percent correct for cognate and noncognate stimuli in Experiment 1.}\label{OOC_latencies}
\end{center}

\end{table}


\section{Discussion}
The goal of the first experiment was to replicate the decontextualized processing effects that support the theory of nonselective word recognition for bilingual speakers. Native English speakers that learned Spanish as a second language  participated in the experiment. They named words that were cognates between Spanish and English and noncognates controls. Analysis of cognate words and noncognate words showed that the cognate words were named faster compared to the control words, suggesting that the bilingual participants were activating both languages.  If the bilinguals had been activating only Spanish, then there would be no reason for the words containing cross-language overlap with English to be read faster. 

%This experiment provides support for the parallel activation of cognate words, though it did lack a monolingual control group.  Despite this shortcoming, a wealth of previous research finds cognate effects and suggests that  cognate effect is likely legitimate. Obtaining the cognate effect out of context is important for the next experiment which investigates word recognition in sentence context.  

%In this experiment, native English speakers who were proficient in Spanish named cognate words faster than noncognate control words. This result suggests that the bilinguals were activating both English and Spanish while reading Spanish words. 

One shortcoming of this experiment is that it lacked a monolingual control group. While great care was taken during the process of matching cognate words to controls, not everything can be perfectly controlled. Thus, it is possible that monolingual speakers could show the same ``cognate effects'' if the stimuli were not sufficiently controlled for lexical properties.  From this experiment alone, there is no way to rule out this possibility and no way to ensure that the cognate effect is actually due to parallel activation. Data collection is being carried out with monolingual English speakers naming the English translations of the target words in order to provide this control comparison. 

For the current investigation, I assume that the cognate effect demonstrated here is real. This assumption is supported by previous studies (reviewed in Section \ref{Intro::OOC}, all of which find cognate effects under similar circumstances (bilinguals recognizing L2 words). Furthermore, the in context control experiments in Chapter \ref{SEInContext} show that monolingual speakers do not exhibit a cognate effect like the bilinguals do.

In sum, the stimuli chosen for the present investigation have been shown to be sufficiently able to elicit cognate effects. In the next chapter, these target words will be embedded in sentences to investigate the role of context and language-specific syntactic constraints on word recognition. 
