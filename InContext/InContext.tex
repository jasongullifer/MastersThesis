\chapter{Experiment 2: Bilinguals in sentence context}\label{SEInContext}
The first experiment demonstrated that the selected materials are sufficient to show that bilingual speakers activate words in both languages in parallel outside of sentence context. The goal of this experiment is to explore bilingual word recognition in sentence context. Specifically, it will replicate previous studies that find evidence for cross-language parallel activation in a sentence context, and it will investigate the extent to which language-specific syntax lessens coactivation. 

For this experiment, sentence contexts were constructed around the target words used in the first experiment. All of the sentences were low semantic constraint, but they varied in whether they contained language-specific syntax or syntax that was language-general. If, like previous research suggests, bilinguals activate lexical candidates in both languages while reading a unilingual sentence, then bilingual participants should recognize cognates more quickly than noncognate controls. However, if language-specific syntax functions to reduce activation of the unintended language, the cognate effect should disappear or be reduced following language-specific material, resulting in an interaction between cognate status and syntactic specificity. 

Two monolingual control groups (English and Spanish) will ensure that any cognate effects are due to bilingualism and not to uncontrolled properties of the stimuli. Because monolinguals know only one language, they should exhibit no difference in recognition between cognate and noncognate words. Additionally, monolinguals can provide a window into whether there are differences between the way sentence containing specific syntax and sentences with language-general syntax cause the target words to be processed.   

\section{Participants}
Three groups of participants were recruited for this experiment: Spanish-English bilinguals, English monolinguals, and Spanish monolinguals. Sixteen Spanish-English bilinguals were recruited from the Pennsylvania State University. They were compensated \$10 per hour for their participation. One bilingual participant was removed because of technical difficulties, leaving 15 remaining participants. Fourteen English monolinguals from the same university were recruited and were given course credit. Two monolinguals were removed due to low accuracy on comprehension questions, leaving 12 monolingual English speakers. Twelve Spanish monolinguals were recruited from the University of Granada and were given course credit for participation. Four of the Spanish monolinguals were removed from the analyses due to technical difficulties, leaving a total of eight Spanish monolingual participants.

\section{Materials}
This experiment included materials in Spanish and English. Bilinguals saw sentences in both languages while monolinguals saw only saw sentences in their language. The 128 Spanish target words (64 cognates and 64 matched noncognate controls) from Experiment 1 as well as their translations into English were used as targets in this experiment. This yielded a  total set of 256 critical items across both languages. For each Spanish target word, two sentences were written. One sentence contained syntax that was specific to Spanish, and the other contained syntax that was language-general. Both sentences were low semantic constraint. The Spanish sentences were translated into English such that all English sentences contained syntax that could apply to either Spanish or English. 

\subsection{Bilingual materials}
Spanish specific syntax was operationalized as the use of both a proclitic and \textit{pro-drop} prior to the point at which the target word appeared. The proclitic is a special type of pronoun not present in English. The proclitic in these sentences redundantly coreferred to the target word. \textit{Pro-drop} is the term for dropping the subject of a clause; in the case of these sentences, the subject of a relative clause was dropped. Like proclitics, \textit{pro-drop} is not present in English. An illustration of a sentence with Spanish-specific syntax is presented in  \ref{specific_example}. The proclitic (``le'') and \textit{pro-drop} (``[pro]'') are marked in bold, and the target word (``profesor'' / ``professor'', a cognate) is italicized. 

\ex.\label{specific_example} Los estudiantes \textbf{le} contaron el cuento que \textbf{[pro]} leyeron el otro d\'{i}a al \textit{profesor} de literatura inglesa.\\
  The students recounted the story that they read the other day to the \textit{professor} of English literature

 Spanish-nonspecific syntax is exemplified in \ref{nonspecific_example}. Note that in this sentence, there is no \textit{pro-drop} nor are there proclitics occurring before the target word.

\ex.\label{nonspecific_example} El taxista que estaba estacionado en la esquina de la panadería llev\'{o} al \textit{profesor} a su casa.\\
  The taxi driver who was parked at the corner of the bakery took the \textit{professor} to her house.


The English version of the materials contained translations of each of the Spanish sentences. The translation was done in such a way that the grammatical structure was nonspecific to English or Spanish; they contained language-general syntax. This was done as a control manipulation to ensure that if there was an interaction between cognate status and syntactic specificity in Spanish that it was due to the syntax and not to arbitrary properties of the sentences.  The translations are exemplified in \ref{specific_example} and \ref{nonspecific_example}, and the full set of stimuli can be found in Appendix \ref{Appendix::InContext}. Note that the general word order is the same across Spanish and English. 

Overall, the materials contained three factors of interest: language (English or Spanish), syntactic specificity (nonspecific or specific), and cognate status (noncognate or cognate) for a total of eight conditions with 64 trials per condition for a total of 512  sentences. These materials were then counterbalanced into two lists within each language. Each list contained 32 trials per condition. In a given list, participants saw  each cognate and its matched control in the same specificity condition (e.g., \textit{cable} and \textit{chispa} [spark] in specific). In the sister list for the other language, the participant would see the same cognate and noncognate but in the opposite specificity condition (e.g., \textit{cable} and \textit{spark} [chispa] in nonspecific).  The order of the presentation of language blocks was counterbalanced across participants. 

The method of counterbalancing in this experiment ensured that a given participant never saw the same sentence within or across languages. However, they did see each target word twice (once in each language). While repetition may cause some degree of priming, it was necessary to in order to have adequate power.  It also allowed for a controlled comparison of cognate and matched noncognate controls within each condition. 

In addition to the experimental stimuli, 12 practice and 48 filler sentences were included in each list. These sentences contained a mixture of cognate and noncognate target words and were not controlled for in terms of syntactic specificity. 

\subsection{Monolingual materials}
The English and Spanish monolingual groups saw materials only in their respective languages. Because seeing translations of the same sentences was not an issue with the monolinguals, they could see both  lists of materials. In other words they saw each item in each condition within their language. For example, an English monolingual in the first block  would see \textit{cable} and \textit{spark} in the nonspecific condition. During the second block she would see \textit{cable} and \textit{spark} in the specific condition.  
   For the monolingual speakers, rate of presentation was varied across two blocks (as opposed to language of presentation for the bilinguals). In one block, the participant would see a fast presentation, and on the second they would see a slower presentation. This order was counterbalanced across participants. This yielded a set of three factors of interest: presentation rate (fast or slow), syntactic specificity (nonspecific or specific), and cognate status (noncognate or cognate).

\section{Procedure}
This experiment was conducted across two visits to the lab. Each visit comprised of a different language or presentation rate.  When participants arrived at the lab on the first day, they were asked to read and complete an informed consent form. After signing the informed consent, they were asked to fill out a language history questionnaire. Participants were then seated at a computer and began the set of experiments. They began with the first part of the in context word naming task. After the naming task, they completed  a set of tasks designed to measure individual differences (working memory and executive control) and proficiency. Following participation, participants were compensated for their time. 

Before leaving the lab, participants were invited back for the second visit. A second visit was  completely voluntary. On the second visit to the lab, participants filled out another informed consent form. They then completed part two of the in context experiment. Following the naming experiment, participants completed the English and Spanish grammar tests. They were again paid for their participation.

Instructions for the in context word naming task were given  verbally and were displayed on the computer screen prior to the start of the task. Participants were told that they would be reading sentences word-by-word and were to name the red target word aloud as quickly and accurately as possible. Before each sentence was presented, a fixation cross (``+'') was displayed. In order to see the sentence, participants were told to press a key. 

Each word of the sentence was presented at a fixed rate. For bilinguals, words were presented at 300 ms. Monolinguals received two presentation rates (150 ms or 300 ms). The entire session was recorded so accuracy could be coded, and a voice trigger was used to record naming latency. Overall, participants spent about two hours in the lab. 

\section{Results}
Data from the RSVP  task were analyzed by comparing cognate latencies to noncognate latencies within each syntactic constraint condition for each language. This procedure was also conducted on naming accuracy. The RSVP data were analyzed separately for each group: Spanish-English bilinguals, English monolinguals, and Spanish monolinguals. Before the results for the RSVP task are shown, the language history and individual difference data will be presented to allow for group comparisons. 

\subsection{Language history and individual difference data}
Data regarding the participants' background and individual difference measures (Simon task and Operation Span) are shown in Table \ref{InContext_lhq_id}. A univariate ANOVA revealed that there were group differences in terms of  age (\textit{F}(2,34) = 6.25, \textit{p} $<$ 0.01). Pairwise comparisons corrected using the TukeyHSD method showed that participants in the bilingual group were older  (\textit{M} $=$ 24.1) than participants in the two monolingual groups (\textit{M}$_{English} =$ 19.6 and \textit{M}$_{Spanish} =$ 19; bilingual vs. English: \textit{p} $<$ 0.05; bilingual vs. Spanish: \textit{p} $<$ 0.05; English vs. Spanish: \textit{p} $>$ 0.05). The groups did not differ significantly  in terms of their Operation Span score, though ANOVA did approach significance (\textit{F}(2,34) = 3.11, \textit{p} $=$ 0.057).  There were no differences between the three groups in terms of the Simon score (\textit{F}(2,34) $<$ 1). 

%Pairwise comparisons using the TukeyHSD method revealed that the Spanish monolingual group has a lower Operation Span (\textit{M} $=$ 36.1) compared to the bilingual group (\textit{M} $=$ 42.6) and the English monolingual group (\textit{M} $=$ 46.3; Spanish vs. English: \textit{p} $<$ 0.05; English vs. bilingual and Spanish vs. Bilingual: \textit{p} $>$ 0.10).

% latex.table(x = as.matrix(lhq), file = "lhq") 
%
\begin{table}[hptb]
\begin{center}
\begin{tabular}{|l|l|l|l|} \hline
\multicolumn{1}{|l|}{}&\multicolumn{1}{l|}{Bilinguals}&\multicolumn{1}{l|}{Eng Monolinguals}&\multicolumn{1}{l|}{Span Monolinguals}\\ \hline
N~~~~~~~~~~~~~~~~~~~~~~~~~~~&15~~~~~~~~~~~&12~~~~~~~~~~~~&8~~~~~~~~~~~\\ 
Age~(years)~~~~~~~~~~~~~~~~~&24.1~(5.2)~~~&19.9~(3.8)~~~~&19~(1.9)~~~~\\ 
Simon~Score~~~~~~~~~~~~~~~~~&46.1~(17.0)~~&46.3~(27.4)~~~&53.7~(14.4)~\\ 
Operation~Span~(Out~of~60)~~&42.6~(10.6)~~&46.3~(7.3)~~~~&36.6~(7.9)~~\\ 
Picture~Naming~Accuracy~~~~~&87\%~(14\%)~~&96\%~(2.1\%)~~& NA\\ 
\hline
\end{tabular}
\caption{Language background data and individual difference measures participants in Experiment 2}\label{InContext_lhq_id}
\end{center}
\end{table}


Average self-assessed language ratings are shown in Table \ref{InContext_lr}. These ratings were analyzed across groups and within groups. Cross group comparisons provided a window into the differences between language proficiency for each group. Comparisons within groups allowed for the assessment of relative language dominance of each group. 

A series of univariate ANOVAs were conducted for the across-group comparisons. Ratings within each language were used as the dependent variable and speaker group was the independent variable. Pairwise comparisons were conducted between each group using the TukeyHSD method. The three groups differed in terms of their English ratings (\textit{F}(2,34) $=$ 49.86, \textit{p} $<$ 0.01). Pairwise comparisons showed that the Spanish monolinguals rated themselves lower in English (\textit{M} $=$ 4.4) than the English monolinguals (\textit{M} $=$ 9.7; Spanish vs. English: \textit{p} $<$ 0.01). The Spanish monolinguals also rated themselves lower compared to the bilinguals (\textit{M} $=$ 8.6, \textit{p} $<$ 0.01). The difference in English ratings between the bilinguals and the English monolinguals was marginally significant (bilingual vs. English: \textit{p} $=$ 0.058). 

% latex.table(x = as.matrix(lr), file = "lr") 
%
\begin{table}[hptb]
\begin{center}
\begin{tabular}{|l|l|l|} \hline
\multicolumn{1}{|l|}{Self assessed language ratings (out of 10)}&\multicolumn{1}{l|}{English}&\multicolumn{1}{l|}{Spanish}\\ \hline
Spanish-English~Bilinguals&8.6~(1.6)~~~~~~~~~~~~~~~~~&9.7~(0.5)~~~~~~~~~~~~~~~~~\\ 
English~Monolinguals~~~~~~&9.7~(0.4)~~~~~~~~~~~~~~~~~&3.1~(1.8)~~~~~~~~~~~~~~~~~\\ 
Spanish~Monolinguals~~~~~~&4.4~(1.4)~~~~~~~~~~~~~~~~~&9.3~(0.6)~~~~~~~~~~~~~~~~~\\ 
\hline
\end{tabular}
\caption{Self-assessed language ratings for participants in Experiment 2}\label{InContext_lr}
\end{center}
\end{table} 



The speaker groups also differed in their self-assessed ratings of Spanish (\textit{F}(2,34) $=$ 120.79, \textit{p} $<$ 0.01). Pairwise comparisons revealed that both the bilinguals and the Spanish monolinguals rated themselves higher in Spanish compared to the English monolinguals (\textit{M}$_{bilinguals} =$ 9.7; \textit{M}$_{English} =$ 3.1; \textit{M}$_{Spanish} =$ 9.4; Bilinguals vs. Spanish: \textit{p} $<$ 0.01; Spanish vs. English: \textit{p} $<$ 0.01). The bilinguals did not differ from the Spanish monolinguals in their Spanish ratings (\textit{p} $>$ 0.05).
 
For within group comparisons, a t-test was conducted for each group comparing English to Spanish ratings. The English monolinguals rated themselves as significantly more proficient in English (\textit{M} $=$ 9.7) than in Spanish, or another L2, (\textit{M} $=$ 3.1; \textit{t}(13) $=$ 13.90, \textit{p} $<$ 0.01). The Spanish monolinguals rated themselves as significantly more proficient in Spanish (\textit{M} $=$ 9.4) than in English, or another L2, (\textit{M} $=$ 4.6; \textit{t}(7) $=$ 9.11, \textit{p} $<$ 0.01). The Spanish-English bilinguals rated themselves as more proficient in Spanish (\textit{M} $=$ 9.7) than in English (\textit{M} $=$ 8.6; \textit{t}(14) $=$ 3.79, \textit{p} $<$ 0.01).

\subsection{RSVP}
Before the data from the RSVP experiment were analyzed, they were trimmed of outliers. Absolute cutoffs were set at 200 ms and 2000 ms, and trials outside of that range were removed from further analyses. Mean reaction times were calculated by participant by condition for trials on which the target word was named correctly. Trials with reaction times 2.5 \textit{SD} above or below this mean were marked as relative outliers and were removed from subsequent analyses. This outlier procedure resulted in the removal of about 4\% of the data. Means and accuracies were then calculated by participant for cognates and noncognates within each specificity condition for each language or timing condition.  The results for each of the monolingual control studies are presented first. Following these results, data from the bilingual RSVP task are reviewed. 

\subsubsection{English monolinguals}
The latency and accuracy data from the English monolinguals in sentence context were subjected to a 2x2x2 (timing x syntactic specificity x cognate status) repeated measures ANOVA. For the latency measure, the analysis revealed that there was a main effect of syntactic specificity (\textit{F}(1,11) $=$ 16.22, \textit{p} $<$ 0.01). Words were named more quickly in the specific condition compared to the nonspecific condition (\textit{M}$_{nonspecific} =$ 613 ms; M$_{specific} =$ 589ms). No other main effects were significant (\textit{F}s $<$ 1). No interactions were significant (timing X syntactic specificity \textit{F}(1,11) $=$ 2.23, \textit{p} $>$ 0.05;  timing X cognate status: \textit{F}(1,11) $<$ 1; syntactic specificity X cognate status: \textit{F}(1,11) $=$ 2.72, \textit{p} $>$ 0.05; timing X syntactic specificity X cognate status: \textit{F}(1,11) $<$ 1). Mean reaction times are shown in Table \ref{emono_means}.

\begin{table}[hptb]
\begin{center}
\begin{tabular}{|c|c|c|c|} \hline
\multicolumn{1}{|c|}{Condition}&\multicolumn{1}{c|}{Mean RT (in ms)}&\multicolumn{1}{c|}{Std. Deviation}&\multicolumn{1}{c|}{N}\\ \hline\hline
150ms~Nonspecific~Cognate~~~~&608.36~~~&~81.10~~~&14~\\ 
150ms~Nonspecific~Noncognate~&619.81~~~&~82.93~~~&14~\\ 
\hline
150ms~Specific~Cognate~~~~~~~&596.20~~~&~79.37~~~&14~\\ 
150ms~Specific~Noncognate~~~~&581.25~~~&~63.55~~~&14~\\ 
\hline\hline
300ms~Nonspecific~Cognate~~~~&611.57~~~&101.27~~~&14~\\ 
300ms~Nonspecific~Noncognate~&612.88~~~&100.09~~~&14~\\ 
\hline
300ms~Specific~Cognate~~~~~~~&595.57~~~&100.52~~~&14~\\ 
300ms~Specific~Noncognate~~~~&583.18~~~&~76.30~~~&14~\\ 
\hline
\end{tabular}
\caption{Mean naming latencies (in ms) in context for English monolingual participants}\label{emono_means}
\end{center}
\end{table}



%accuracy
The analysis on the accuracy measure showed that there were no significant effects of timing, specificity, or cognate status and no interactions between them (timing: \textit{F}(1,11) $<$ 1; specificity:  \textit{F}(1,11) $<$ 1;  cognate status:  \textit{F}(1,11) $=$ 1.40, \textit{p} $>$ 0.05; timing X specificity:  \textit{F}(1,11) $<$ 1;  timing X cognate status:  \textit{F}(1,11) $=$ 1.60 \textit{p} $>$ 0.05 ; specificity X cognate status:  \textit{F}(1,11) $<$ 1; timing X specificity X cognate status:  \textit{F}(1,11) $=$ 2, \textit{p} $>$ 0.05).

Overall the results for the English monolinguals indicate that there may be some  differences in terms of the English sentence structure between the specific and nonspecific condition that influences word recognition. Importantly, there were no significant effects of cognate status, suggesting that the cognates and noncognate controls were well matched lexically.

%bias towards the facilititory cognate effect in the English materials. However, the effect was very small (under 10ms) and  it only surfaced in the nonspecific sentences.  Hence, this effect may not be meaningful. 

\subsubsection{Spanish monolinguals}
The accuracy and latency data from the Spanish monolinguals in sentence context were subjected to a 2x2x2 (timing x syntactic specificity x cognate status) repeated measures ANOVA. In the latency data, no main effects were significant (all \textit{F}s $<$ 1). The interaction between timing and specificity approached significance (\textit{F}(1,7) $=$ 3.20, \textit{p} $=$ 0.12). No other interactions approached significance (\textit{F}s $<$ 1). Mean reaction times are presented in Table \ref{smono_means}.

% latex.table(x = as.matrix(smono_means), file = "smono_means") 
%
\begin{table}[hptb]
\begin{center}
\begin{tabular}{|c|c|c|c|} \hline
\multicolumn{1}{|c|}{Condition}&\multicolumn{1}{c|}{Mean RT (in ms)}&\multicolumn{1}{c|}{Std. Deviation}&\multicolumn{1}{c|}{N}\\ \hline\hline
150ms~Nonspecific~Cognate~~~~&667.33~~~&111.023~~~&8~\\ 
150ms~Nonspecific~Noncognate~&672.38~~~&~86.390~~~&8~\\ 
\hline
150ms~Specific~Cognate~~~~~~~&643.26~~~&~68.185~~~&8~\\ 
150ms~Specific~Noncognate~~~~&641.93~~~&~91.554~~~&8~\\ 
\hline\hline
300ms~Nonspecific~Cognate~~~~&654.00~~~&~78.606~~~&8~\\ 
300ms~Nonspecific~Noncognate~&649.88~~~&~91.322~~~&8~\\ 
\hline
300ms~Specific~Cognate~~~~~~~&682.47~~~&~94.602~~~&8~\\ 
300ms~Specific~Noncognate~~~~&673.55~~~&~83.213~~~&8~\\ 
\hline
\end{tabular}
\caption{Mean naming latencies (in ms) in context for Spanish monolingual participants}\label{smono_means}
\end{center}
\end{table}



Because the interaction between timing and specificity would likely become significant if more participants were tested, simple effects analyses were conducted. For the fast presentation rate (150 ms), there was a marginal main effect of syntactic specificity (\textit{F}(1,7) $=$ 5.30, \textit{p} $=$ 0.055) with the reaction time for words named in the specific condition faster than in the nonspecific condition (\textit{M}$_{nonspecific} =$ 670 ms, \textit{M}$_{specific} =$ 643 ms). No other main effects or interactions were significant (\textit{F}s $<$ 1). In the slower presentation block (300 ms), there were no significant main effects or interactions (\textit{F}s $<$ 1). 

%accuracy
The analysis on the accuracy measure revealed that there was a marginal main effect of timing and a marginal interaction between syntactic specificity and cognate status (timing: \textit{F}(1,7) $=$ 5.362, \textit{p} $=$ 0.054; specificity X cognate status: \textit{F}(1,7) $=$ 5.119, \textit{p} $=$ 0.058). No other main effects or interactions approached significance (specificity: \textit{F}(1,7) $=$  1.12, \textit{p} $>$ 0.05 ; cognate status:  \textit{F}(1,7) $<$ 1 ; timing X spec: \textit{F}(1,7) $=$ 2.14, \textit{p} $>$ 0.05; timing X cognate status: \textit{F}(1,7) $=$ 2.59, \textit{p} $>$ 0.05; timing X specificity X cognate status: \textit{F}(1,7) $<$ 1). 

Though the interaction was not significant, a simple effects analysis was conducted in order to get a sense of the nature of the effect. The data were collapsed across timing, because it did not interact with other factors, and an analysis of the effect of cognate status at each level of specificity was conducted. In the nonspecific condition, there was a marginal effect of cognate status (\textit{F}(1,7) $=$ 4.20, \textit{p} $=$ 0.080), with cognates having 1.8\% higher accuracy than noncognate words. In the specific condition, there was no effect of cognate status (\textit{F}(1,7) $=$ 1.84, \textit{p} $>$ 0.10).

The results of the Spanish monolinguals show that the Spanish materials were well controlled. There were no significant effects of cognate status (though cognates may be named more accurately in the nonspecific condition). There was a hint of an effect of specificity, indicating that the Spanish monolinguals may process the sentences with Spanish specific syntax differentially compared to the nonspecific sentences. 

\subsubsection{Spanish-English bilinguals}
A 2x2x2 repeated measures ANOVA was conducted on the latency and accuracy data from the Spanish-English bilinguals. The factors included in the model were language (English or Spanish), syntactic specificity (nonspecific or specific), and cognate status (cognate or noncognate). For the latency data (shown in Table \ref{overall_means}), there was a significant main effect of cognate status (\textit{F}(1,14) $=$ 20.22, \textit{p} $<$ 0.01), indicating that cognates were named faster than noncognates (\textit{M}$_{cognates} =$ 694 ms, \textit{M}$_{noncognates} =$ 718 ms). No other main effects were significant (\textit{F}s $<$ 1). No interactions were significant (language X specificity: \textit{F}(1,14) $< 1$; language X cognate status: \textit{F}(1,14) $< 1$; specificity X cognate status: \textit{F}(1,14) $= 1.35$, \textit{p} $> .05$; language X specificity X cognate status: \textit{F}(1,14) $= 1.22$, \textit{p} $> .05$). 

\begin{table}[hptb]
\begin{center}
\begin{tabular}{|c|c|c|c|} \hline
\multicolumn{1}{|c|}{Condition}&\multicolumn{1}{c|}{Mean RT (in ms)}&\multicolumn{1}{c|}{Std..Deviation}&\multicolumn{1}{c|}{N}\\ \hline\hline
English~Nonspecific~Cognate~~~~&705.31~~~&180.844~~~&15~\\ 
English~Nonspecific~Noncognate~&720.84~~~&176.136~~~&15~\\ 
\hline
English~Specific~Cognate~~~~~~~&689.61~~~&157.824~~~&15~\\ 
English~Specific~Noncognate~~~~&729.08~~~&196.690~~~&15~\\ 
\hline\hline
Spanish~Nonspecific~Cognate~~~~&691.87~~~&183.502~~~&15~\\ 
Spanish~Nonspecific~Noncognate~&711.91~~~&183.809~~~&15~\\ 
\hline
Spanish~Specific~Cognate~~~~~~~&689.53~~~&165.934~~~&15~\\ 
Spanish~Specific~Noncognate~~~~&713.33~~~&190.399~~~&15~\\ 
\hline
\end{tabular}
\caption{Mean naming latencies (in ms) for all Spanish-English bilingual participants}\label{overall_means}
\end{center}
\end{table}


%accuracy
In the accuracy data, there was a marginal main effect of language (\textit{F}(1,14) $=$ 3.76, \textit{p} $=$ 0.73), hinting that Spanish may be more accurate than English (\textit{M}$_{Spanish} =$ 98\%; \textit{M}$_{English} =$ 96\%). No other effects or interactions were significant (specificity: \textit{F}(1,14) $<$ 1 ; cognate status: \textit{F}(1,14) $=$ 1.50, \textit{p} $>$ 0.05; language X specificity: \textit{F}(1,14) $=$ 2.52, \textit{p} $>$ 0.05; language X cognate status: \textit{F}(1,14) $<$ 1; specificity X cognate status: \textit{F}(1,14) $<$ 1; language X specificity X cognate status: \textit{F}(1,14) $<$ 1). 

In summary, the bilinguals showed a facilitatory cognate effect in the latency measures in both Spanish and English blocks. This cognate effect did not appear to interact with language of the task or specificity. Neither monolingual control group showed evidence of cognate effects (though the Spanish monolinguals were marginally more accurate on cognate words). Both groups of monolinguals did show effects of the specificity manipulation. Monolingual English speakers were faster in the specific condition. Monolingual Spanish speakers were likely faster in the specific condition, but only under the faster timing manipulation. This difference was not significant, but the nonsignificance is likely due to a lack of power with only eight participants. 

\section{Discussion}
The present experiment demonstrated that bilinguals, but not monolinguals, activated both languages in parallel while reading sentences in one language alone. This parallel activation was not constrained by the presence of language-specific syntax in all speakers. The monolingual participants did show evidence of differential processing in the language-specific syntax condition compared to the nonspecific condition. 

%The differential processing in the specific conditions suggests that the monolinguals were sensitive to the syntactic differences between specific and nonspecific sentences in English and Spanish. When a group of Spanish-English bilinguals who showed recognition patterns similar to the Spanish monolinguals (recognizing words faster for specific sentences) were analyzed seperatly, evidence for a reduction of the cognate effect under specific sentences in the Spanish block began to emerge, though it was only marginally significant. This reduction did not occur in the English block. 

Spanish-English bilinguals named cognates faster than noncognates when these words were embedded in sentence context. Facilitation was not observable for monolingual speakers reading the same sentences. Together these results indicate that the cognate facilitation for the bilinguals is not due to lexical factors. Instead, the effects are due to the presence of cross-language overlap in form and meaning for the cognates. These  effects would only be present if bilinguals were activating both languages in parallel. The findings of this experiment converge with those of many previous studies \parencite[e.g.,][]{Baten2010,Chambers2009, Duyck2007, Libben2009, Schwartz2006, VanAssche2009, VanAssche2010, VanHell2008}, all of which find that the presence of a sentence context alone is not sufficient to constrain word recognition to a single language.

In order to examine the role of language-specific syntax, a subset of the sentences included in this experiment incorporated two language-specific features: (1) proclitics, a special type of pronoun that refers to the target noun phrase and (2) pro-drop, the dropping of the subject of a clause, in this case an object-relative clause. Neither of these features are present in English syntax, making them unique to Spanish when only the two languages are considered. Pro-drop is an extremely salient cross-linguistic cue. Some languages have pro-drop and other do not, and attempting to pro-drop in a languages that does not do allow for it, causes   a sentence to become ungrammatical (e.g., ``John likes books. *Is reading one now''). In the case of the object relative clauses in these sentences, pro-drop is required, otherwise the sentence becomes ungrammatical. The proclitic is also not present in English. In the present materials, the proclitic always coreferred with the target cognate or noncognate, establishing tightly bound relationship with the target. One might predict that these two features would be a red flag as to the language of the sentence and that they may allow for language-selective access. However, this was not the case.

In the sentences containing language-specific syntax, the cognate effect persisted, and its magnitude was not changed. Hence, parallel activation of the two languages was not constrained by language-specific syntactic constraints. This finding is compatible with BIA+ model \parencite[][]{Dijkstra2002} and with researchers who posit a limited role for linguistic factors such as syntax or semantics in word recognition \parencite[e.g.,][]{Desmet2007}. Word recognition in the BIA+ model is a bottom-up process. Expectations of the language of the sentence  do not influence word recognition in this model. If language-specific syntax in the present study were able to enhance  expectations about the language of the task, then it did not come into play inside the word recognition system. Either the bilinguals were not sensitive to the manipulation, or they were sensitive but the manipulation had no effect. However, there is evidence from the monolingual participants that the syntax specific condition can influence word recognition. In the 150 ms timing condition, Spanish monolingual were facilitated in the specific syntax condition.  

%However, a subset of the speakers did show a hint of a modulation of the cognate effect in sentences containing Spanish-specific syntax. These were the bilinguals who displayed a pattern of speeding up in the sentences containing specific syntax, a pattern that was also shown by the Spanish monolinguals. The bilinguals who did not speed up in the specific syntax condition, showed a cognate effect in all sentences. Hence, it appears that the bilinguals who behaved like monolinguals in the utilization of the specific syntax, may have been able to switch off the other language following language-specific syntax. A major question that stems from this finding are what factors influence whether speakers, bilingual or monolingual, allow them to process the specific syntax differently.   

%The bilinguals all saw the same, slower presentation rate (300ms), yet some bilinguals showed the speed-up for this timing. Because, bilinguals are known to process sentences more slowly in either language compared to monolinguals,  one possibility for this difference is that, for the bilinguals, the 300ms timing  is more comparable to the 150ms timing for the monolingual speakers. However, without having a rate manipulation for the bilinguals, this hypothesis cannot be explicitly tested in the current experiment. 

%A series of post-hoc correlations were conducted to investigate other potentially relevant factors for determining whether a bilingual will behave like the monolingual English speakers. Two a priori hypotheses were that higher working memory and higher comprehension question accuracies allow a speaker to exploit the language-specific syntax like monolingual speakers. However, there was no evidence for these hypotheses in the correlational analysis.  No other measures were correlated with the utilization of the specific syntax.

In sum, the results of this experiment confirm that bilingual word recognition is a fundamentally nonselective process. This finding falls in line with a the findings from a plethora of previous studies \parencite[e.g.,][]{Baten2010,Chambers2009,Dijkstra1998,Dijkstra1999,Duyck2007,Grainger1992,Gollan1997,Jared2001,Libben2009,Marian2003,Schwartz2006,Schwartz2007,VanAssche2009,VanAssche2010,VanHell2002,VanHell2008}. It would seem that the elimination of parallel activation of the unintended language is not easy. Thus far, only two factors have been found that allow for selective access during word recognition in context: a highly biased semantic constraint \parencite[e.g.,][]{Libben2009,Chambers2009,Schwartz2006,VanHell2008}, and non-overlapping word classes across languages \parencite[e.g.,][]{Baten2010,Sunderman2006}.

%The current results  show that for a subset of speaker, syntactic constraints may be able to eliminate parallel activation. However this conclusion is very tentative, and more participants must be tested to verify the claim. 

%had both languages active, and this cognate effect occurred in both the Spanish and English blocks. Neither the Spanish monolinguals nor the English monolinguals showed  cognate effect, suggesting that the cognate and noncognate words were well matched on lexical properties in English and Spanish. Taken together, these results suggest that the cognate effect for the Spanish-English bilinguals is due to their status as bilingual speakers, and not to uncontrolled lexical or sentential aspects of the materials. 

%A somewhat similar pattern was found for English-monolinguals; they showed a very small cognate effect in the nonspecific condition only. This pattern  likely due to unintentional effects of lexical or sentential characteristics. Importantly, the cognate effect shown by the bilinguals is probably not completely due to lexical or sentential effects.

%The cognate effects in the bilingual group can be attributed to the participants bilingualism for three reasons. First, the cognate effect for the bilinguals is numerically larger (25ms effect) than the 5ms cognate effect that the English monolinguals exhibited (which though significant, may not even be meaningful). This suggests that the bilinguals experienced increased facilitation for the cognate words compared to the monolinguals\footnote{The increase in cognate effect size could also be attributed to the bilinguals being slower at the English naming task compared (\textit{M} = 711ms) to the English monolinguals (\textit{M} = 601ms; \textit{t}(19.19) = 2.21, p < 0.05).}. Secondly, the cognate effect for the bilinguals did not with syntactic specificity, as it did for the monolingual speakers. The monolingual speakers showed no effect of cognate status in the specific syntax conditions, but the bilinguals did.  Taken together, these results suggest that the cognate effect demonstrated by the bilinguals is likely due to the cognate status and not completely to extraneous properties of the stimuli.  
