\chapter{General directions and methodologies}\label{Roadmap}
There were two main goals to this study. The first goal was to further confirm that a sentence context is not enough by itself to constrain word recognition to one language. The second goal of the study was to assess whether the use of language-specific syntactic constraints in sentence contexts is sufficient to allow for language-selective word recognition. To investigate these issues, participants performed a Rapid Serial Visual Presentation (RSVP) experiment (Chapter \ref{SEInContext}) in which they read sentences presented word-by-word and named target words aloud. The target words were embedded in sentence contexts that contained syntax specific to Spanish or contained language-general syntax.  Control studies ensured that any effects are due to the intended manipulations and not to idiosyncratic properties of the stimuli. A variety of cognitive tasks and language-proficiency tasks were implemented to insure that our groups of participants were matched for language experience and cognitive performance. These  measures were also  used to explore the ways in which individual differences influence processing. The current chapter provides an overview of the participants recruited and the experimental procedures used in each part of study.

\section{Participants}\label{Roadmap::Participants}
Four groups of participants were recruited for this investigation. Bilinguals who were native speakers of English and acquired Spanish as a second language (English-Spanish bilinguals) were recruited for the out-of context naming study in Chapter \ref{ESOOC}. Native speakers of Spanish who learned English as a second language (Spanish-English bilinguals) participated in the RSVP experiment in Chapter \ref{SEInContext}. Two monolingual control groups were  recruited for the in context control experiment: a group of English monolinguals and a group of Spanish monolinguals. The Spanish monolinguals were recruited from the University of Granada in Spain, while all other groups were recruited from the Pennsylvania State University in the United States. 

The learners of Spanish studying at Penn State University were recruited for the out of context task in Chapter \ref{ESOOC} in order to ensure that our cognates were sensitive to parallel activation of two languages. Because the size of the cognate effect is typically larger for participants using their second language, a failure to find effect in the second language would indicate that effects would not likely be found for participants reading in their native language. 

Native speakers of Spanish were necessary for the main experiment in Chapter \ref{SEInContext} because our critical manipulation depends on speakers' knowledge of and ability to process complex syntactic structures in Spanish. Hence, a  high proficiency in Spanish was required. While high proficiency L2 speakers certainly exist, some researchers argue that L2 speakers do not have access to certain syntactic structures in the L2 \parencite[e.g.,][]{Clahsen2006}. While I remain agnostic towards this claim, the choice of native Spanish speakers ensures that they will be able to process the structures that the critical manipulation depends on. 

In order to ensure that any effects found were due to the intended manipulations, two groups of monolinguals were recruited to participate in each of the language blocks. A group of Spanish monolinguals recruited from the University of Granada, Spain read the Spanish portion of the in context study. A group of English monolinguals from Penn State were recruited to read the English portion of the materials. Because monolinguals only know one language, they should not demonstrate any effects due to cognate status. 

\section{Word naming tasks}
The word naming tasks were used to assess the degree of cross-language activation. Participants were presented with cognate and control words, and the time it took them to begin naming was recorded.  The latencies for cognates and noncognates were compared to measure parallel activation. Two types of word naming tasks were used in the present set of experiments: out of context word naming, and word naming in sentence context. Before detailing each of these tasks, I review how the target words were selected. 

\subsection{Selection of target words}
A set of 64 cognates between English and Spanish were selected. Each of the cognates was matched to a Spanish noncognate control word, yielding a total of 128 target words. The targets were matched for  word frequency \parencite[e.g.,][]{Alameda1995}, length, number of syllables, number of phonemes, first phoneme, and animacy. Cognates and controls were not always perfectly matched for every measure, but importance was placed on frequency and length. Neither length nor frequency differed significantly between Spanish cognate and noncognate words (both \textit{t}(126) $< 1$). These  target words were  then translated into English. The English cognate and control materials are reproduced in tables Table \ref{eng_cognates} and Table \ref{eng_controls}. Again, neither length nor frequency differed significantly between the cognate and control words. The cognate materials and control stimuli are reproduced in Appendix \ref{Appendix::OOC}.

%% \begin{centering}
%% % latex.table(x = as.matrix(cogs), file = "cogs", longtable = T) 
%
\setlongtables
\begin{longtable}{|l|c|c|c|c|c|c|}
\hline
\multicolumn{1}{|c|}{Cognate}&\multicolumn{1}{c|}{Frequency}&\multicolumn{1}{c|}{First.Phoneme}&\multicolumn{1}{c|}{Syllables}&\multicolumn{1}{c|}{Phonemes}&\multicolumn{1}{c|}{Length}&\multicolumn{1}{c|}{Animacy}\\ \hline
\endhead
\hline\endfoot
bus~~~~~~~~~~&~~2~~~~~~~~~~&b~~~~~~~~~~~~&1~~~~~~~~~~~~&~3~~~~~~~~~~~&~3~~~~~~~~~~~&i~~~~~~~~~~~~\\ 
general~~~~~~&632~~~~~~~~~~&x~~~~~~~~~~~~&3~~~~~~~~~~~~&~7~~~~~~~~~~~&~7~~~~~~~~~~~&a~~~~~~~~~~~~\\ 
colegas~~~~~~&~56~~~~~~~~~~&k~~~~~~~~~~~~&3~~~~~~~~~~~~&~7~~~~~~~~~~~&~7~~~~~~~~~~~&a~~~~~~~~~~~~\\ 
garaje~~~~~~~&~22~~~~~~~~~~&g~~~~~~~~~~~~&3~~~~~~~~~~~~&~6~~~~~~~~~~~&~6~~~~~~~~~~~&i~~~~~~~~~~~~\\ 
cable~~~~~~~~&~16~~~~~~~~~~&k~~~~~~~~~~~~&2~~~~~~~~~~~~&~5~~~~~~~~~~~&~5~~~~~~~~~~~&i~~~~~~~~~~~~\\ 
proyecto~~~~~&155~~~~~~~~~~&p~~~~~~~~~~~~&3~~~~~~~~~~~~&~8~~~~~~~~~~~&~8~~~~~~~~~~~&i~~~~~~~~~~~~\\ 
c\'{a}mara~~~~~~~&114~~~~~~~~~~&k~~~~~~~~~~~~&3~~~~~~~~~~~~&~6~~~~~~~~~~~&~6~~~~~~~~~~~&i~~~~~~~~~~~~\\ 
turistas~~~~~&~51~~~~~~~~~~&t~~~~~~~~~~~~&3~~~~~~~~~~~~&~8~~~~~~~~~~~&~8~~~~~~~~~~~&a~~~~~~~~~~~~\\ 
jirafa~~~~~~~&~~2~~~~~~~~~~&x~~~~~~~~~~~~&3~~~~~~~~~~~~&~6~~~~~~~~~~~&~6~~~~~~~~~~~&a~~~~~~~~~~~~\\ 
reportero~~~~&~~3~~~~~~~~~~&r~~~~~~~~~~~~&4~~~~~~~~~~~~&~9~~~~~~~~~~~&~9~~~~~~~~~~~&a~~~~~~~~~~~~\\ 
plato~~~~~~~~&~85~~~~~~~~~~&p~~~~~~~~~~~~&2~~~~~~~~~~~~&~5~~~~~~~~~~~&~5~~~~~~~~~~~&i~~~~~~~~~~~~\\ 
pirata~~~~~~~&~12~~~~~~~~~~&p~~~~~~~~~~~~&3~~~~~~~~~~~~&~6~~~~~~~~~~~&~6~~~~~~~~~~~&a~~~~~~~~~~~~\\ 
pipa~~~~~~~~~&~38~~~~~~~~~~&p~~~~~~~~~~~~&2~~~~~~~~~~~~&~4~~~~~~~~~~~&~4~~~~~~~~~~~&i~~~~~~~~~~~~\\ 
planta~~~~~~~&~89~~~~~~~~~~&p~~~~~~~~~~~~&2~~~~~~~~~~~~&~6~~~~~~~~~~~&~6~~~~~~~~~~~&a~~~~~~~~~~~~\\ 
profesora~~~~&~15~~~~~~~~~~&p~~~~~~~~~~~~&4~~~~~~~~~~~~&~9~~~~~~~~~~~&~9~~~~~~~~~~~&a~~~~~~~~~~~~\\ 
estatua~~~~~~&~36~~~~~~~~~~&E~~~~~~~~~~~~&3~~~~~~~~~~~~&~7~~~~~~~~~~~&~7~~~~~~~~~~~&i~~~~~~~~~~~~\\ 
cliente~~~~~~&~40~~~~~~~~~~&k~~~~~~~~~~~~&2~~~~~~~~~~~~&~7~~~~~~~~~~~&~7~~~~~~~~~~~&a~~~~~~~~~~~~\\ 
cobra~~~~~~~~&~37~~~~~~~~~~&k~~~~~~~~~~~~&2~~~~~~~~~~~~&~5~~~~~~~~~~~&~5~~~~~~~~~~~&a~~~~~~~~~~~~\\ 
cubo~~~~~~~~~&~26~~~~~~~~~~&k~~~~~~~~~~~~&2~~~~~~~~~~~~&~4~~~~~~~~~~~&~4~~~~~~~~~~~&i~~~~~~~~~~~~\\ 
organizador~~&~~4~~~~~~~~~~&O~~~~~~~~~~~~&5~~~~~~~~~~~~&11~~~~~~~~~~~&11~~~~~~~~~~~&a~~~~~~~~~~~~\\ 
viol\'{i}n~~~~~~~&~17~~~~~~~~~~&b~~~~~~~~~~~~&2~~~~~~~~~~~~&~6~~~~~~~~~~~&~6~~~~~~~~~~~&i~~~~~~~~~~~~\\ 
c\'{i}rculo~~~~~~&~59~~~~~~~~~~&T~~~~~~~~~~~~&3~~~~~~~~~~~~&~7~~~~~~~~~~~&~7~~~~~~~~~~~&i~~~~~~~~~~~~\\ 
pistola~~~~~~&~50~~~~~~~~~~&p~~~~~~~~~~~~&3~~~~~~~~~~~~&~7~~~~~~~~~~~&~7~~~~~~~~~~~&i~~~~~~~~~~~~\\ 
oficial~~~~~~&118~~~~~~~~~~&O~~~~~~~~~~~~&3~~~~~~~~~~~~&~7~~~~~~~~~~~&~7~~~~~~~~~~~&a~~~~~~~~~~~~\\ 
problemas~~~~&279~~~~~~~~~~&p~~~~~~~~~~~~&3~~~~~~~~~~~~&~9~~~~~~~~~~~&~9~~~~~~~~~~~&i~~~~~~~~~~~~\\ 
computadora~~&~~9~~~~~~~~~~&k~~~~~~~~~~~~&5~~~~~~~~~~~~&11~~~~~~~~~~~&11~~~~~~~~~~~&i~~~~~~~~~~~~\\ 
detective~~~~&~12~~~~~~~~~~&d~~~~~~~~~~~~&4~~~~~~~~~~~~&~9~~~~~~~~~~~&~9~~~~~~~~~~~&a~~~~~~~~~~~~\\ 
atleta~~~~~~~&~~8~~~~~~~~~~&A~~~~~~~~~~~~&3~~~~~~~~~~~~&~6~~~~~~~~~~~&~6~~~~~~~~~~~&a~~~~~~~~~~~~\\ 
compositor~~~&~~4~~~~~~~~~~&k~~~~~~~~~~~~&4~~~~~~~~~~~~&10~~~~~~~~~~~&10~~~~~~~~~~~&a~~~~~~~~~~~~\\ 
coronel~~~~~~&~~6~~~~~~~~~~&k~~~~~~~~~~~~&3~~~~~~~~~~~~&~7~~~~~~~~~~~&~7~~~~~~~~~~~&a~~~~~~~~~~~~\\ 
paciente~~~~~&~52~~~~~~~~~~&p~~~~~~~~~~~~&3~~~~~~~~~~~~&~8~~~~~~~~~~~&~8~~~~~~~~~~~&a~~~~~~~~~~~~\\ 
hamburguesa~~&~~3~~~~~~~~~~&A~~~~~~~~~~~~&4~~~~~~~~~~~~&~9~~~~~~~~~~~&11~~~~~~~~~~~&i~~~~~~~~~~~~\\ 
capitales~~~~&~13~~~~~~~~~~&k~~~~~~~~~~~~&4~~~~~~~~~~~~&~9~~~~~~~~~~~&~9~~~~~~~~~~~&i~~~~~~~~~~~~\\ 
sopa~~~~~~~~~&~31~~~~~~~~~~&s~~~~~~~~~~~~&2~~~~~~~~~~~~&~4~~~~~~~~~~~&~4~~~~~~~~~~~&i~~~~~~~~~~~~\\ 
vendedor~~~~~&~21~~~~~~~~~~&b~~~~~~~~~~~~&3~~~~~~~~~~~~&~8~~~~~~~~~~~&~8~~~~~~~~~~~&a~~~~~~~~~~~~\\ 
decisi\'{o}n~~~~~&140~~~~~~~~~~&d~~~~~~~~~~~~&4~~~~~~~~~~~~&~8~~~~~~~~~~~&~8~~~~~~~~~~~&i~~~~~~~~~~~~\\ 
rata~~~~~~~~~&~34~~~~~~~~~~&r~~~~~~~~~~~~&2~~~~~~~~~~~~&~4~~~~~~~~~~~&~4~~~~~~~~~~~&a~~~~~~~~~~~~\\ 
su\'{e}ter~~~~~~~&~~6~~~~~~~~~~&s~~~~~~~~~~~~&2~~~~~~~~~~~~&~6~~~~~~~~~~~&~6~~~~~~~~~~~&i~~~~~~~~~~~~\\ 
ingeniero~~~~&~70~~~~~~~~~~&I~~~~~~~~~~~~&4~~~~~~~~~~~~&~9~~~~~~~~~~~&~9~~~~~~~~~~~&a~~~~~~~~~~~~\\ 
beb\'{e}~~~~~~~~~&~30~~~~~~~~~~&b~~~~~~~~~~~~&2~~~~~~~~~~~~&~4~~~~~~~~~~~&~4~~~~~~~~~~~&a~~~~~~~~~~~~\\ 
%\newpage
instituto~~~~&~84~~~~~~~~~~&I~~~~~~~~~~~~&4~~~~~~~~~~~~&~9~~~~~~~~~~~&~9~~~~~~~~~~~&i~~~~~~~~~~~~\\ 
tanque~~~~~~~&~12~~~~~~~~~~&t~~~~~~~~~~~~&2~~~~~~~~~~~~&~5~~~~~~~~~~~&~6~~~~~~~~~~~&i~~~~~~~~~~~~\\ 
director~~~~~&173~~~~~~~~~~&d~~~~~~~~~~~~&3~~~~~~~~~~~~&~8~~~~~~~~~~~&~8~~~~~~~~~~~&a~~~~~~~~~~~~\\ 
estrategia~~~&~54~~~~~~~~~~&E~~~~~~~~~~~~&4~~~~~~~~~~~~&10~~~~~~~~~~~&10~~~~~~~~~~~&i~~~~~~~~~~~~\\ 
caf\'{e}~~~~~~~~~&210~~~~~~~~~~&k~~~~~~~~~~~~&2~~~~~~~~~~~~&~4~~~~~~~~~~~&~4~~~~~~~~~~~&i~~~~~~~~~~~~\\ 
catedral~~~~~&~58~~~~~~~~~~&k~~~~~~~~~~~~&3~~~~~~~~~~~~&~8~~~~~~~~~~~&~8~~~~~~~~~~~&i~~~~~~~~~~~~\\ 
tel\'{e}fono~~~~~&186~~~~~~~~~~&t~~~~~~~~~~~~&4~~~~~~~~~~~~&~8~~~~~~~~~~~&~8~~~~~~~~~~~&i~~~~~~~~~~~~\\ 
carpintero~~~&~12~~~~~~~~~~&k~~~~~~~~~~~~&4~~~~~~~~~~~~&10~~~~~~~~~~~&10~~~~~~~~~~~&a~~~~~~~~~~~~\\ 
brócoli~~~~~~&~~1~~~~~~~~~~&b~~~~~~~~~~~~&0~~~~~~~~~~~~&~0~~~~~~~~~~~&~7~~~~~~~~~~~&a~~~~~~~~~~~~\\ 
caramelos~~~~&~15~~~~~~~~~~&k~~~~~~~~~~~~&4~~~~~~~~~~~~&~9~~~~~~~~~~~&~9~~~~~~~~~~~&i~~~~~~~~~~~~\\ 
familia~~~~~~&495~~~~~~~~~~&f~~~~~~~~~~~~&3~~~~~~~~~~~~&~7~~~~~~~~~~~&~7~~~~~~~~~~~&a~~~~~~~~~~~~\\ 
presidente~~~&138~~~~~~~~~~&p~~~~~~~~~~~~&4~~~~~~~~~~~~&10~~~~~~~~~~~&10~~~~~~~~~~~&a~~~~~~~~~~~~\\ 
estudiante~~~&~37~~~~~~~~~~&E~~~~~~~~~~~~&4~~~~~~~~~~~~&10~~~~~~~~~~~&10~~~~~~~~~~~&a~~~~~~~~~~~~\\ 
recepcionista&~~3~~~~~~~~~~&r~~~~~~~~~~~~&5~~~~~~~~~~~~&11~~~~~~~~~~~&13~~~~~~~~~~~&a~~~~~~~~~~~~\\ 
sof\'{a}~~~~~~~~~&~67~~~~~~~~~~&s~~~~~~~~~~~~&2~~~~~~~~~~~~&~4~~~~~~~~~~~&~4~~~~~~~~~~~&i~~~~~~~~~~~~\\ 
bi\'{o}loga~~~~~~&~~1~~~~~~~~~~&b~~~~~~~~~~~~&4~~~~~~~~~~~~&~7~~~~~~~~~~~&~7~~~~~~~~~~~&a~~~~~~~~~~~~\\ 
presentador~~&~~3~~~~~~~~~~&p~~~~~~~~~~~~&4~~~~~~~~~~~~&11~~~~~~~~~~~&11~~~~~~~~~~~&a~~~~~~~~~~~~\\ 
artista~~~~~~&102~~~~~~~~~~&A~~~~~~~~~~~~&3~~~~~~~~~~~~&~7~~~~~~~~~~~&~7~~~~~~~~~~~&a~~~~~~~~~~~~\\ 
cereal~~~~~~~&~~1~~~~~~~~~~&T~~~~~~~~~~~~&3~~~~~~~~~~~~&~6~~~~~~~~~~~&~6~~~~~~~~~~~&i~~~~~~~~~~~~\\ 
dinamita~~~~~&~~2~~~~~~~~~~&d~~~~~~~~~~~~&4~~~~~~~~~~~~&~8~~~~~~~~~~~&~8~~~~~~~~~~~&i~~~~~~~~~~~~\\ 
autoridades~~&~37~~~~~~~~~~&A~~~~~~~~~~~~&0~~~~~~~~~~~~&~0~~~~~~~~~~~&11~~~~~~~~~~~&a~~~~~~~~~~~~\\ 
miembros~~~~~&140~~~~~~~~~~&m~~~~~~~~~~~~&0~~~~~~~~~~~~&~0~~~~~~~~~~~&~8~~~~~~~~~~~&a~~~~~~~~~~~~\\ 
ant\'{i}lope~~~~~&~~1~~~~~~~~~~&A~~~~~~~~~~~~&4~~~~~~~~~~~~&~8~~~~~~~~~~~&~8~~~~~~~~~~~&a~~~~~~~~~~~~\\ 
canguro~~~~~~&~~2~~~~~~~~~~&k~~~~~~~~~~~~&3~~~~~~~~~~~~&~7~~~~~~~~~~~&~7~~~~~~~~~~~&a~~~~~~~~~~~~\\ 
\hline
\caption{List of cognate target words.}\label{span_cognates}
\end{longtable}

%% % latex.table(x = as.matrix(controls), file = "controls", longtable = T) 
%
\setlongtables
\begin{longtable}{|l|c|c|c|c|c|c|}

\hline
\multicolumn{1}{|c|}{Control}&\multicolumn{1}{c|}{Frequency}&\multicolumn{1}{c|}{First.Phoneme}&\multicolumn{1}{c|}{Syllables}&\multicolumn{1}{c|}{Phonemes}&\multicolumn{1}{c|}{Length}&\multicolumn{1}{c|}{Animacy}\\ \hline
\endhead
\hline\endfoot
laca~~~~~~~~~&~~2~~~~~~~~~~&l~~~~~~~~~~~~&2~~~~~~~~~~~~&~4~~~~~~~~~~~&~4~~~~~~~~~~~&i~~~~~~~~~~~~\\ 
escritura~~~~&~73~~~~~~~~~~&E~~~~~~~~~~~~&4~~~~~~~~~~~~&~9~~~~~~~~~~~&~9~~~~~~~~~~~&i~~~~~~~~~~~~\\ 
cuaderno~~~~~&~55~~~~~~~~~~&k~~~~~~~~~~~~&3~~~~~~~~~~~~&~8~~~~~~~~~~~&~8~~~~~~~~~~~&i~~~~~~~~~~~~\\ 
manejo~~~~~~~&~24~~~~~~~~~~&m~~~~~~~~~~~~&3~~~~~~~~~~~~&~6~~~~~~~~~~~&~6~~~~~~~~~~~&i~~~~~~~~~~~~\\ 
chispa~~~~~~~&~19~~~~~~~~~~&J~~~~~~~~~~~~&2~~~~~~~~~~~~&~5~~~~~~~~~~~&~6~~~~~~~~~~~&i~~~~~~~~~~~~\\ 
barrio~~~~~~~&161~~~~~~~~~~&b~~~~~~~~~~~~&2~~~~~~~~~~~~&~5~~~~~~~~~~~&~6~~~~~~~~~~~&i~~~~~~~~~~~~\\ 
primavera~~~~&114~~~~~~~~~~&p~~~~~~~~~~~~&4~~~~~~~~~~~~&~9~~~~~~~~~~~&~9~~~~~~~~~~~&i~~~~~~~~~~~~\\ 
herida~~~~~~~&~52~~~~~~~~~~&E~~~~~~~~~~~~&3~~~~~~~~~~~~&~5~~~~~~~~~~~&~6~~~~~~~~~~~&i~~~~~~~~~~~~\\ 
bistec~~~~~~~&~~2~~~~~~~~~~&b~~~~~~~~~~~~&2~~~~~~~~~~~~&~6~~~~~~~~~~~&~6~~~~~~~~~~~&i~~~~~~~~~~~~\\ 
pescadora~~~~&~~1~~~~~~~~~~&p~~~~~~~~~~~~&4~~~~~~~~~~~~&~9~~~~~~~~~~~&~9~~~~~~~~~~~&a~~~~~~~~~~~~\\ 
torre~~~~~~~~&~85~~~~~~~~~~&t~~~~~~~~~~~~&2~~~~~~~~~~~~&~4~~~~~~~~~~~&~5~~~~~~~~~~~&i~~~~~~~~~~~~\\ 
folio~~~~~~~~&~13~~~~~~~~~~&f~~~~~~~~~~~~&2~~~~~~~~~~~~&~5~~~~~~~~~~~&~5~~~~~~~~~~~&i~~~~~~~~~~~~\\ 
perro~~~~~~~~&~38~~~~~~~~~~&p~~~~~~~~~~~~&2~~~~~~~~~~~~&~4~~~~~~~~~~~&~5~~~~~~~~~~~&a~~~~~~~~~~~~\\ 
hierro~~~~~~~&~89~~~~~~~~~~&j~~~~~~~~~~~~&2~~~~~~~~~~~~&~4~~~~~~~~~~~&~6~~~~~~~~~~~&i~~~~~~~~~~~~\\ 
lavado~~~~~~~&~13~~~~~~~~~~&l~~~~~~~~~~~~&3~~~~~~~~~~~~&~6~~~~~~~~~~~&~6~~~~~~~~~~~&i~~~~~~~~~~~~\\ 
espuma~~~~~~~&~36~~~~~~~~~~&E~~~~~~~~~~~~&3~~~~~~~~~~~~&~6~~~~~~~~~~~&~6~~~~~~~~~~~&i~~~~~~~~~~~~\\ 
postre~~~~~~~&~41~~~~~~~~~~&p~~~~~~~~~~~~&2~~~~~~~~~~~~&~6~~~~~~~~~~~&~6~~~~~~~~~~~&i~~~~~~~~~~~~\\ 
cinta~~~~~~~~&~37~~~~~~~~~~&T~~~~~~~~~~~~&2~~~~~~~~~~~~&~5~~~~~~~~~~~&~5~~~~~~~~~~~&i~~~~~~~~~~~~\\ 
ni\~{n}ez~~~~~~~~&~26~~~~~~~~~~&n~~~~~~~~~~~~&2~~~~~~~~~~~~&~5~~~~~~~~~~~&~5~~~~~~~~~~~&i~~~~~~~~~~~~\\ 
impresora~~~~&~~4~~~~~~~~~~&I~~~~~~~~~~~~&4~~~~~~~~~~~~&~9~~~~~~~~~~~&~9~~~~~~~~~~~&i~~~~~~~~~~~~\\ 
bragueta~~~~~&~18~~~~~~~~~~&b~~~~~~~~~~~~&3~~~~~~~~~~~~&~7~~~~~~~~~~~&~8~~~~~~~~~~~&i~~~~~~~~~~~~\\ 
castigo~~~~~~&~59~~~~~~~~~~&k~~~~~~~~~~~~&3~~~~~~~~~~~~&~7~~~~~~~~~~~&~7~~~~~~~~~~~&i~~~~~~~~~~~~\\ 
corbata~~~~~~&~51~~~~~~~~~~&k~~~~~~~~~~~~&3~~~~~~~~~~~~&~7~~~~~~~~~~~&~7~~~~~~~~~~~&i~~~~~~~~~~~~\\ 
despacho~~~~~&118~~~~~~~~~~&d~~~~~~~~~~~~&3~~~~~~~~~~~~&~7~~~~~~~~~~~&~8~~~~~~~~~~~&i~~~~~~~~~~~~\\ 
alma~~~~~~~~~&329~~~~~~~~~~&A~~~~~~~~~~~~&2~~~~~~~~~~~~&~4~~~~~~~~~~~&~4~~~~~~~~~~~&i~~~~~~~~~~~~\\ 
escalerilla~~&~~9~~~~~~~~~~&E~~~~~~~~~~~~&5~~~~~~~~~~~~&10~~~~~~~~~~~&11~~~~~~~~~~~&i~~~~~~~~~~~~\\ 
cabalgata~~~~&~~8~~~~~~~~~~&k~~~~~~~~~~~~&4~~~~~~~~~~~~&~9~~~~~~~~~~~&~9~~~~~~~~~~~&i~~~~~~~~~~~~\\ 
avestruz~~~~~&~~8~~~~~~~~~~&A~~~~~~~~~~~~&3~~~~~~~~~~~~&~8~~~~~~~~~~~&~8~~~~~~~~~~~&a~~~~~~~~~~~~\\ 
congelador~~~&~~4~~~~~~~~~~&k~~~~~~~~~~~~&4~~~~~~~~~~~~&10~~~~~~~~~~~&10~~~~~~~~~~~&i~~~~~~~~~~~~\\ 
b\'{a}scula~~~~~~&~~7~~~~~~~~~~&b~~~~~~~~~~~~&3~~~~~~~~~~~~&~7~~~~~~~~~~~&~7~~~~~~~~~~~&i~~~~~~~~~~~~\\ 
agujero~~~~~~&~53~~~~~~~~~~&A~~~~~~~~~~~~&4~~~~~~~~~~~~&~7~~~~~~~~~~~&~7~~~~~~~~~~~&i~~~~~~~~~~~~\\ 
hermanastro~~&~~3~~~~~~~~~~&E~~~~~~~~~~~~&4~~~~~~~~~~~~&10~~~~~~~~~~~&11~~~~~~~~~~~&a~~~~~~~~~~~~\\ 
mendigo~~~~~~&~13~~~~~~~~~~&m~~~~~~~~~~~~&3~~~~~~~~~~~~&~7~~~~~~~~~~~&~7~~~~~~~~~~~&a~~~~~~~~~~~~\\ 
lomo~~~~~~~~~&~31~~~~~~~~~~&l~~~~~~~~~~~~&2~~~~~~~~~~~~&~4~~~~~~~~~~~&~4~~~~~~~~~~~&i~~~~~~~~~~~~\\ 
encuesta~~~~~&~21~~~~~~~~~~&E~~~~~~~~~~~~&3~~~~~~~~~~~~&~8~~~~~~~~~~~&~8~~~~~~~~~~~&i~~~~~~~~~~~~\\ 
fiesta~~~~~~~&140~~~~~~~~~~&f~~~~~~~~~~~~&2~~~~~~~~~~~~&~6~~~~~~~~~~~&~6~~~~~~~~~~~&i~~~~~~~~~~~~\\ 
viajero~~~~~~&~33~~~~~~~~~~&b~~~~~~~~~~~~&3~~~~~~~~~~~~&~7~~~~~~~~~~~&~7~~~~~~~~~~~&a~~~~~~~~~~~~\\ 
biombo~~~~~~~&~~5~~~~~~~~~~&b~~~~~~~~~~~~&2~~~~~~~~~~~~&~6~~~~~~~~~~~&~6~~~~~~~~~~~&i~~~~~~~~~~~~\\ 
crecimiento~~&~70~~~~~~~~~~&k~~~~~~~~~~~~&4~~~~~~~~~~~~&11~~~~~~~~~~~&11~~~~~~~~~~~&i~~~~~~~~~~~~\\ 
cueva~~~~~~~~&~29~~~~~~~~~~&k~~~~~~~~~~~~&2~~~~~~~~~~~~&~5~~~~~~~~~~~&~5~~~~~~~~~~~&i~~~~~~~~~~~~\\ 
%\newpage
papa~~~~~~~~~&~84~~~~~~~~~~&p~~~~~~~~~~~~&2~~~~~~~~~~~~&~4~~~~~~~~~~~&~4~~~~~~~~~~~&i~~~~~~~~~~~~\\ 
harina~~~~~~~&~12~~~~~~~~~~&A~~~~~~~~~~~~&3~~~~~~~~~~~~&~5~~~~~~~~~~~&~6~~~~~~~~~~~&i~~~~~~~~~~~~\\ 
actuaci\'{o}n~~~~&~53~~~~~~~~~~&A~~~~~~~~~~~~&4~~~~~~~~~~~~&~9~~~~~~~~~~~&~9~~~~~~~~~~~&i~~~~~~~~~~~~\\ 
encargado~~~~&~55~~~~~~~~~~&E~~~~~~~~~~~~&4~~~~~~~~~~~~&~9~~~~~~~~~~~&~9~~~~~~~~~~~&a~~~~~~~~~~~~\\ 
belleza~~~~~~&212~~~~~~~~~~&b~~~~~~~~~~~~&3~~~~~~~~~~~~&~6~~~~~~~~~~~&~7~~~~~~~~~~~&i~~~~~~~~~~~~\\ 
ascensor~~~~~&~55~~~~~~~~~~&A~~~~~~~~~~~~&3~~~~~~~~~~~~&~8~~~~~~~~~~~&~8~~~~~~~~~~~&i~~~~~~~~~~~~\\ 
caballo~~~~~~&187~~~~~~~~~~&k~~~~~~~~~~~~&3~~~~~~~~~~~~&~6~~~~~~~~~~~&~7~~~~~~~~~~~&a~~~~~~~~~~~~\\ 
calcetines~~~&~26~~~~~~~~~~&k~~~~~~~~~~~~&4~~~~~~~~~~~~&10~~~~~~~~~~~&10~~~~~~~~~~~&i~~~~~~~~~~~~\\ 
arbitro~~~~~~&~~2~~~~~~~~~~&A~~~~~~~~~~~~&0~~~~~~~~~~~~&~0~~~~~~~~~~~&~7~~~~~~~~~~~&a~~~~~~~~~~~~\\ 
cabellera~~~~&~15~~~~~~~~~~&k~~~~~~~~~~~~&4~~~~~~~~~~~~&~8~~~~~~~~~~~&~9~~~~~~~~~~~&i~~~~~~~~~~~~\\ 
ni\~{n}os~~~~~~~~&497~~~~~~~~~~&-1~~~~~~~~~~~&0~~~~~~~~~~~~&~0~~~~~~~~~~~&~5~~~~~~~~~~~&a~~~~~~~~~~~~\\ 
amiga~~~~~~~~&136~~~~~~~~~~&A~~~~~~~~~~~~&3~~~~~~~~~~~~&~5~~~~~~~~~~~&~5~~~~~~~~~~~&a~~~~~~~~~~~~\\ 
extranjeros~~&~40~~~~~~~~~~&-1~~~~~~~~~~~&0~~~~~~~~~~~~&~0~~~~~~~~~~~&11~~~~~~~~~~~&a~~~~~~~~~~~~\\ 
guardabosques&~~1~~~~~~~~~~&g~~~~~~~~~~~~&4~~~~~~~~~~~~&12~~~~~~~~~~~&13~~~~~~~~~~~&a~~~~~~~~~~~~\\ 
muro~~~~~~~~~&~72~~~~~~~~~~&m~~~~~~~~~~~~&2~~~~~~~~~~~~&~4~~~~~~~~~~~&~4~~~~~~~~~~~&i~~~~~~~~~~~~\\ 
cabrito~~~~~~&~~1~~~~~~~~~~&k~~~~~~~~~~~~&3~~~~~~~~~~~~&~7~~~~~~~~~~~&~7~~~~~~~~~~~&a~~~~~~~~~~~~\\ 
bibliotecario&~~4~~~~~~~~~~&b~~~~~~~~~~~~&7~~~~~~~~~~~~&11~~~~~~~~~~~&13~~~~~~~~~~~&a~~~~~~~~~~~~\\ 
ciegos~~~~~~~&~73~~~~~~~~~~&T~~~~~~~~~~~~&2~~~~~~~~~~~~&~5~~~~~~~~~~~&~5~~~~~~~~~~~&a~~~~~~~~~~~~\\ 
duraznos~~~~~&~~1~~~~~~~~~~&-1~~~~~~~~~~~&0~~~~~~~~~~~~&~0~~~~~~~~~~~&~8~~~~~~~~~~~&a~~~~~~~~~~~~\\ 
ciruelas~~~~~&~~3~~~~~~~~~~&T~~~~~~~~~~~~&3~~~~~~~~~~~~&~8~~~~~~~~~~~&~8~~~~~~~~~~~&a~~~~~~~~~~~~\\ 
hu\'{e}spedes~~~~&~37~~~~~~~~~~&-1~~~~~~~~~~~&0~~~~~~~~~~~~&~0~~~~~~~~~~~&~9~~~~~~~~~~~&a~~~~~~~~~~~~\\ 
edificio~~~~~&141~~~~~~~~~~&E~~~~~~~~~~~~&4~~~~~~~~~~~~&~8~~~~~~~~~~~&~8~~~~~~~~~~~&i~~~~~~~~~~~~\\ 
cachorro~~~~~&~~1~~~~~~~~~~&k~~~~~~~~~~~~&3~~~~~~~~~~~~&~6~~~~~~~~~~~&~7~~~~~~~~~~~&a~~~~~~~~~~~~\\ 
duendes~~~~~~&~~1~~~~~~~~~~&-1~~~~~~~~~~~&0~~~~~~~~~~~~&~0~~~~~~~~~~~&~7~~~~~~~~~~~&a~~~~~~~~~~~~\\ 
\hline
\caption{List of noncognate control target words}\label{span_controls}
\end{longtable}

%% %% latex.table(x = as.matrix(en_cognates), file = "en_cognates",      longtable = T) 
%
\setlongtables
\begin{longtable}{|c|c|c|c|c|c|c|}
\hline
\multicolumn{1}{|c|}{Cognate}&\multicolumn{1}{c|}{Frequency}&\multicolumn{1}{c|}{First Phoneme}&\multicolumn{1}{c|}{Syllables}&\multicolumn{1}{c|}{Phonemes}&\multicolumn{1}{c|}{Length}&\multicolumn{1}{c|}{Animacy}\\ \hline
\endhead
\hline\endfoot
bus~~~~~~~~~&~34~~~~~~~~~&b~~~~~~~~~~~&1~~~~~~~~~~~&~3~~~~~~~~~~&~3~~~~~~~~~~&i~~~~~~~~~~~\\ 
general~~~~~&497~~~~~~~~~&dZ~~~~~~~~~~&2~~~~~~~~~~~&~6~~~~~~~~~~&~7~~~~~~~~~~&a~~~~~~~~~~~\\ 
colleagues~~&~23~~~~~~~~~&k~~~~~~~~~~~&2~~~~~~~~~~~&~6~~~~~~~~~~&10~~~~~~~~~~&a~~~~~~~~~~~\\ 
garage~~~~~~&~21~~~~~~~~~&g~~~~~~~~~~~&2~~~~~~~~~~~&~5~~~~~~~~~~&~6~~~~~~~~~~&i~~~~~~~~~~~\\ 
cable~~~~~~~&~~7~~~~~~~~~&k~~~~~~~~~~~&2~~~~~~~~~~~&~4~~~~~~~~~~&~5~~~~~~~~~~&i~~~~~~~~~~~\\ 
project~~~~~&~93~~~~~~~~~&p~~~~~~~~~~~&2~~~~~~~~~~~&~7~~~~~~~~~~&~7~~~~~~~~~~&i~~~~~~~~~~~\\ 
camera~~~~~~&~36~~~~~~~~~&k~~~~~~~~~~~&2~~~~~~~~~~~&~5~~~~~~~~~~&~6~~~~~~~~~~&i~~~~~~~~~~~\\ 
tourists~~~~&~12~~~~~~~~~&t~~~~~~~~~~~&2~~~~~~~~~~~&~7~~~~~~~~~~&~8~~~~~~~~~~&a~~~~~~~~~~~\\ 
giraffe~~~~~&~~0~~~~~~~~~&dZ~~~~~~~~~~&2~~~~~~~~~~~&~5~~~~~~~~~~&~7~~~~~~~~~~&a~~~~~~~~~~~\\ 
reporter~~~~&~20~~~~~~~~~&r~~~~~~~~~~~&3~~~~~~~~~~~&~7~~~~~~~~~~&~8~~~~~~~~~~&a~~~~~~~~~~~\\ 
plate~~~~~~~&~22~~~~~~~~~&p~~~~~~~~~~~&1~~~~~~~~~~~&~4~~~~~~~~~~&~5~~~~~~~~~~&i~~~~~~~~~~~\\ 
pirate~~~~~~&~~4~~~~~~~~~&p~~~~~~~~~~~&2~~~~~~~~~~~&~5~~~~~~~~~~&~6~~~~~~~~~~&a~~~~~~~~~~~\\ 
pipe~~~~~~~~&~20~~~~~~~~~&p~~~~~~~~~~~&1~~~~~~~~~~~&~3~~~~~~~~~~&~4~~~~~~~~~~&i~~~~~~~~~~~\\ 
plant~~~~~~~&125~~~~~~~~~&p~~~~~~~~~~~&1~~~~~~~~~~~&~5~~~~~~~~~~&~5~~~~~~~~~~&a~~~~~~~~~~~\\ 
professor~~~&~57~~~~~~~~~&p~~~~~~~~~~~&3~~~~~~~~~~~&~7~~~~~~~~~~&~9~~~~~~~~~~&a~~~~~~~~~~~\\ 
statue~~~~~~&~17~~~~~~~~~&s~~~~~~~~~~~&2~~~~~~~~~~~&~5~~~~~~~~~~&~6~~~~~~~~~~&i~~~~~~~~~~~\\ 
client~~~~~~&~15~~~~~~~~~&k~~~~~~~~~~~&1~~~~~~~~~~~&~6~~~~~~~~~~&~6~~~~~~~~~~&a~~~~~~~~~~~\\ 
cobra~~~~~~~&~~3~~~~~~~~~&k~~~~~~~~~~~&2~~~~~~~~~~~&~5~~~~~~~~~~&~5~~~~~~~~~~&a~~~~~~~~~~~\\ 
cube~~~~~~~~&~~1~~~~~~~~~&k~~~~~~~~~~~&1~~~~~~~~~~~&~4~~~~~~~~~~&~4~~~~~~~~~~&i~~~~~~~~~~~\\ 
organizer~~~&~~3~~~~~~~~~&O~~~~~~~~~~~&4~~~~~~~~~~~&~8~~~~~~~~~~&~9~~~~~~~~~~&a~~~~~~~~~~~\\ 
violin~~~~~~&~11~~~~~~~~~&v~~~~~~~~~~~&2~~~~~~~~~~~&~6~~~~~~~~~~&~6~~~~~~~~~~&i~~~~~~~~~~~\\ 
circle~~~~~~&~60~~~~~~~~~&s~~~~~~~~~~~&2~~~~~~~~~~~&~4~~~~~~~~~~&~6~~~~~~~~~~&i~~~~~~~~~~~\\ 
pistol~~~~~~&~27~~~~~~~~~&p~~~~~~~~~~~&2~~~~~~~~~~~&~5~~~~~~~~~~&~6~~~~~~~~~~&i~~~~~~~~~~~\\ 
official~~~~&~75~~~~~~~~~&@~~~~~~~~~~~&3~~~~~~~~~~~&~5~~~~~~~~~~&~8~~~~~~~~~~&a~~~~~~~~~~~\\ 
problems~~~~&247~~~~~~~~~&p~~~~~~~~~~~&2~~~~~~~~~~~&~8~~~~~~~~~~&~8~~~~~~~~~~&i~~~~~~~~~~~\\ 
computer~~~~&~13~~~~~~~~~&k~~~~~~~~~~~&3~~~~~~~~~~~&~8~~~~~~~~~~&~8~~~~~~~~~~&i~~~~~~~~~~~\\ 
detective~~~&~52~~~~~~~~~&d~~~~~~~~~~~&3~~~~~~~~~~~&~8~~~~~~~~~~&~9~~~~~~~~~~&a~~~~~~~~~~~\\ 
athlete~~~~~&~~9~~~~~~~~~&a~~~~~~~~~~~&2~~~~~~~~~~~&~5~~~~~~~~~~&~7~~~~~~~~~~&a~~~~~~~~~~~\\ 
composer~~~~&~31~~~~~~~~~&k~~~~~~~~~~~&3~~~~~~~~~~~&~7~~~~~~~~~~&~8~~~~~~~~~~&a~~~~~~~~~~~\\ 
colonel~~~~~&~37~~~~~~~~~&k~~~~~~~~~~~&2~~~~~~~~~~~&~4~~~~~~~~~~&~7~~~~~~~~~~&a~~~~~~~~~~~\\ 
patient~~~~~&~86~~~~~~~~~&p~~~~~~~~~~~&2~~~~~~~~~~~&~5~~~~~~~~~~&~7~~~~~~~~~~&a~~~~~~~~~~~\\ 
hamburger~~~&~~6~~~~~~~~~&h~~~~~~~~~~~&3~~~~~~~~~~~&~7~~~~~~~~~~&~9~~~~~~~~~~&i~~~~~~~~~~~\\ 
capitals~~~~&~~4~~~~~~~~~&k~~~~~~~~~~~&3~~~~~~~~~~~&~7~~~~~~~~~~&~8~~~~~~~~~~&i~~~~~~~~~~~\\ 
soup~~~~~~~~&~16~~~~~~~~~&s~~~~~~~~~~~&1~~~~~~~~~~~&~3~~~~~~~~~~&~4~~~~~~~~~~&i~~~~~~~~~~~\\ 
vendor~~~~~~&~~1~~~~~~~~~&v~~~~~~~~~~~&2~~~~~~~~~~~&~5~~~~~~~~~~&~6~~~~~~~~~~&a~~~~~~~~~~~\\ 
decision~~~~&119~~~~~~~~~&d~~~~~~~~~~~&3~~~~~~~~~~~&~6~~~~~~~~~~&~8~~~~~~~~~~&i~~~~~~~~~~~\\ 
rat~~~~~~~~~&~~6~~~~~~~~~&r~~~~~~~~~~~&1~~~~~~~~~~~&~3~~~~~~~~~~&~3~~~~~~~~~~&a~~~~~~~~~~~\\ 
sweater~~~~~&~14~~~~~~~~~&s~~~~~~~~~~~&2~~~~~~~~~~~&~5~~~~~~~~~~&~7~~~~~~~~~~&i~~~~~~~~~~~\\ 
engineer~~~~&~42~~~~~~~~~&E~~~~~~~~~~~&3~~~~~~~~~~~&~7~~~~~~~~~~&~8~~~~~~~~~~&a~~~~~~~~~~~\\ 
baby~~~~~~~~&~62~~~~~~~~~&b~~~~~~~~~~~&2~~~~~~~~~~~&~4~~~~~~~~~~&~4~~~~~~~~~~&a~~~~~~~~~~~\\ 
institute~~~&~50~~~~~~~~~&I~~~~~~~~~~~&3~~~~~~~~~~~&~8~~~~~~~~~~&~9~~~~~~~~~~&i~~~~~~~~~~~\\ 
tank~~~~~~~~&~12~~~~~~~~~&t~~~~~~~~~~~&1~~~~~~~~~~~&~4~~~~~~~~~~&~4~~~~~~~~~~&i~~~~~~~~~~~\\ 
director~~~~&101~~~~~~~~~&d~~~~~~~~~~~&3~~~~~~~~~~~&~7~~~~~~~~~~&~8~~~~~~~~~~&a~~~~~~~~~~~\\ 
strategy~~~~&~22~~~~~~~~~&s~~~~~~~~~~~&3~~~~~~~~~~~&~8~~~~~~~~~~&~8~~~~~~~~~~&i~~~~~~~~~~~\\ 
coffee~~~~~~&~78~~~~~~~~~&k~~~~~~~~~~~&2~~~~~~~~~~~&~4~~~~~~~~~~&~6~~~~~~~~~~&i~~~~~~~~~~~\\ 
cathedral~~~&~~8~~~~~~~~~&k~~~~~~~~~~~&3~~~~~~~~~~~&~8~~~~~~~~~~&~9~~~~~~~~~~&i~~~~~~~~~~~\\ 
telephone~~~&~76~~~~~~~~~&t~~~~~~~~~~~&3~~~~~~~~~~~&~7~~~~~~~~~~&~9~~~~~~~~~~&i~~~~~~~~~~~\\ 
carpenter~~~&~~6~~~~~~~~~&k~~~~~~~~~~~&3~~~~~~~~~~~&~8~~~~~~~~~~&~9~~~~~~~~~~&a~~~~~~~~~~~\\ 
broccoli~~~~&~~1~~~~~~~~~&b~~~~~~~~~~~&3~~~~~~~~~~~&~7~~~~~~~~~~&~8~~~~~~~~~~&a~~~~~~~~~~~\\ 
caramels~~~~&~~1~~~~~~~~~&k~~~~~~~~~~~&3~~~~~~~~~~~&~7~~~~~~~~~~&~8~~~~~~~~~~&i~~~~~~~~~~~\\ 
family~~~~~~&331~~~~~~~~~&f~~~~~~~~~~~&3~~~~~~~~~~~&~6~~~~~~~~~~&~6~~~~~~~~~~&a~~~~~~~~~~~\\ 
president~~~&382~~~~~~~~~&p~~~~~~~~~~~&3~~~~~~~~~~~&~8~~~~~~~~~~&~9~~~~~~~~~~&a~~~~~~~~~~~\\ 
student~~~~~&131~~~~~~~~~&s~~~~~~~~~~~&2~~~~~~~~~~~&~6~~~~~~~~~~&~7~~~~~~~~~~&a~~~~~~~~~~~\\ 
receptionist&~~5~~~~~~~~~&r~~~~~~~~~~~&4~~~~~~~~~~~&10~~~~~~~~~~&12~~~~~~~~~~&a~~~~~~~~~~~\\ 
sofa~~~~~~~~&~~6~~~~~~~~~&s~~~~~~~~~~~&2~~~~~~~~~~~&~4~~~~~~~~~~&~4~~~~~~~~~~&i~~~~~~~~~~~\\ 
biologist~~~&~~2~~~~~~~~~&b~~~~~~~~~~~&4~~~~~~~~~~~&~9~~~~~~~~~~&~9~~~~~~~~~~&a~~~~~~~~~~~\\ 
presenter~~~&~~1~~~~~~~~~&p~~~~~~~~~~~&3~~~~~~~~~~~&~8~~~~~~~~~~&~9~~~~~~~~~~&a~~~~~~~~~~~\\ 
artist~~~~~~&~57~~~~~~~~~&A~~~~~~~~~~~&2~~~~~~~~~~~&~6~~~~~~~~~~&~6~~~~~~~~~~&a~~~~~~~~~~~\\ 
cereal~~~~~~&~17~~~~~~~~~&s~~~~~~~~~~~&2~~~~~~~~~~~&~6~~~~~~~~~~&~6~~~~~~~~~~&i~~~~~~~~~~~\\ 
dynamite~~~~&~~5~~~~~~~~~&d~~~~~~~~~~~&3~~~~~~~~~~~&~7~~~~~~~~~~&~8~~~~~~~~~~&i~~~~~~~~~~~\\ 
authorities~&~39~~~~~~~~~&@~~~~~~~~~~~&4~~~~~~~~~~~&~8~~~~~~~~~~&11~~~~~~~~~~&a~~~~~~~~~~~\\ 
members~~~~~&325~~~~~~~~~&m~~~~~~~~~~~&2~~~~~~~~~~~&~6~~~~~~~~~~&~7~~~~~~~~~~&a~~~~~~~~~~~\\ 
antelope~~~~&~~7~~~~~~~~~&a~~~~~~~~~~~&3~~~~~~~~~~~&~6~~~~~~~~~~&~8~~~~~~~~~~&a~~~~~~~~~~~\\ 
kangaroo~~~~&~~0~~~~~~~~~&k~~~~~~~~~~~&3~~~~~~~~~~~&~7~~~~~~~~~~&~8~~~~~~~~~~&a~~~~~~~~~~~\\ 
\hline
\caption{List of English cognate words}\label{eng_cognates}
\end{longtable}

%% %% latex.table(x = as.matrix(en_controls), file = "en_controls",      longtable = T) 
%
\setlongtables
\begin{longtable}{|c|c|c|c|c|c|c|}
\hline
\multicolumn{1}{|c|}{Control}&\multicolumn{1}{c|}{Frequency}&\multicolumn{1}{c|}{First Phoneme}&\multicolumn{1}{c|}{Syllables}&\multicolumn{1}{c|}{Phonemes}&\multicolumn{1}{c|}{Length}&\multicolumn{1}{c|}{Animacy}\\ \hline
\endhead
\hline\endfoot
hairspray~~~&~~0~~~~~~~~~&h~~~~~~~~~~~&2~~~~~~~~~~~&7~~~~~~~~~~~&~9~~~~~~~~~~&i~~~~~~~~~~~\\ 
deed~~~~~~~~&~~8~~~~~~~~~&a~~~~~~~~~~~&2~~~~~~~~~~~&7~~~~~~~~~~~&~7~~~~~~~~~~&i~~~~~~~~~~~\\ 
notebook~~~~&~~2~~~~~~~~~&n~~~~~~~~~~~&2~~~~~~~~~~~&6~~~~~~~~~~~&~8~~~~~~~~~~&i~~~~~~~~~~~\\ 
handling~~~~&~38~~~~~~~~~&h~~~~~~~~~~~&3~~~~~~~~~~~&7~~~~~~~~~~~&~8~~~~~~~~~~&i~~~~~~~~~~~\\ 
spark~~~~~~~&~12~~~~~~~~~&s~~~~~~~~~~~&1~~~~~~~~~~~&5~~~~~~~~~~~&~5~~~~~~~~~~&i~~~~~~~~~~~\\ 
neighborhood&~58~~~~~~~~~&n~~~~~~~~~~~&3~~~~~~~~~~~&7~~~~~~~~~~~&12~~~~~~~~~~&i~~~~~~~~~~~\\ 
spring~~~~~~&127~~~~~~~~~&s~~~~~~~~~~~&1~~~~~~~~~~~&5~~~~~~~~~~~&~6~~~~~~~~~~&i~~~~~~~~~~~\\ 
wound~~~~~~~&~28~~~~~~~~~&w~~~~~~~~~~~&1~~~~~~~~~~~&4~~~~~~~~~~~&~5~~~~~~~~~~&i~~~~~~~~~~~\\ 
steak~~~~~~~&~10~~~~~~~~~&s~~~~~~~~~~~&1~~~~~~~~~~~&4~~~~~~~~~~~&~5~~~~~~~~~~&i~~~~~~~~~~~\\ 
fisherwoman~&~~0~~~~~~~~~&f~~~~~~~~~~~&4~~~~~~~~~~~&9~~~~~~~~~~~&11~~~~~~~~~~&a~~~~~~~~~~~\\ 
tower~~~~~~~&~13~~~~~~~~~&t~~~~~~~~~~~&1~~~~~~~~~~~&3~~~~~~~~~~~&~5~~~~~~~~~~&i~~~~~~~~~~~\\ 
report~~~~~~&174~~~~~~~~~&d~~~~~~~~~~~&3~~~~~~~~~~~&9~~~~~~~~~~~&~8~~~~~~~~~~&i~~~~~~~~~~~\\ 
dog~~~~~~~~~&~75~~~~~~~~~&d~~~~~~~~~~~&1~~~~~~~~~~~&3~~~~~~~~~~~&~3~~~~~~~~~~&a~~~~~~~~~~~\\ 
iron~~~~~~~~&~43~~~~~~~~~&a~~~~~~~~~~~&1~~~~~~~~~~~&3~~~~~~~~~~~&~4~~~~~~~~~~&i~~~~~~~~~~~\\ 
wash~~~~~~~~&~37~~~~~~~~~&w~~~~~~~~~~~&1~~~~~~~~~~~&3~~~~~~~~~~~&~4~~~~~~~~~~&i~~~~~~~~~~~\\ 
foam~~~~~~~~&~37~~~~~~~~~&f~~~~~~~~~~~&1~~~~~~~~~~~&3~~~~~~~~~~~&~4~~~~~~~~~~&i~~~~~~~~~~~\\ 
dessert~~~~~&~~7~~~~~~~~~&d~~~~~~~~~~~&2~~~~~~~~~~~&5~~~~~~~~~~~&~7~~~~~~~~~~&i~~~~~~~~~~~\\ 
ribbon~~~~~~&~12~~~~~~~~~&r~~~~~~~~~~~&2~~~~~~~~~~~&5~~~~~~~~~~~&~6~~~~~~~~~~&i~~~~~~~~~~~\\ 
childhood~~~&~50~~~~~~~~~&tS~~~~~~~~~~&2~~~~~~~~~~~&7~~~~~~~~~~~&~9~~~~~~~~~~&i~~~~~~~~~~~\\ 
printer~~~~~&~~3~~~~~~~~~&p~~~~~~~~~~~&2~~~~~~~~~~~&6~~~~~~~~~~~&~7~~~~~~~~~~&i~~~~~~~~~~~\\ 
zipper~~~~~~&~~1~~~~~~~~~&z~~~~~~~~~~~&2~~~~~~~~~~~&4~~~~~~~~~~~&~6~~~~~~~~~~&i~~~~~~~~~~~\\ 
punishment~~&~21~~~~~~~~~&p~~~~~~~~~~~&3~~~~~~~~~~~&9~~~~~~~~~~~&10~~~~~~~~~~&i~~~~~~~~~~~\\ 
tie~~~~~~~~~&~23~~~~~~~~~&t~~~~~~~~~~~&1~~~~~~~~~~~&2~~~~~~~~~~~&~3~~~~~~~~~~&i~~~~~~~~~~~\\ 
workroom~~~~&~~0~~~~~~~~~&A~~~~~~~~~~~&2~~~~~~~~~~~&4~~~~~~~~~~~&~6~~~~~~~~~~&i~~~~~~~~~~~\\ 
soul~~~~~~~~&~47~~~~~~~~~&s~~~~~~~~~~~&1~~~~~~~~~~~&3~~~~~~~~~~~&~4~~~~~~~~~~&i~~~~~~~~~~~\\ 
ladder~~~~~~&~19~~~~~~~~~&l~~~~~~~~~~~&2~~~~~~~~~~~&4~~~~~~~~~~~&~6~~~~~~~~~~&i~~~~~~~~~~~\\ 
parade~~~~~~&~25~~~~~~~~~&p~~~~~~~~~~~&2~~~~~~~~~~~&5~~~~~~~~~~~&~6~~~~~~~~~~&i~~~~~~~~~~~\\ 
ostrich~~~~~&~~0~~~~~~~~~&O~~~~~~~~~~~&2~~~~~~~~~~~&6~~~~~~~~~~~&~7~~~~~~~~~~&a~~~~~~~~~~~\\ 
freezer~~~~~&~~1~~~~~~~~~&f~~~~~~~~~~~&2~~~~~~~~~~~&5~~~~~~~~~~~&~7~~~~~~~~~~&i~~~~~~~~~~~\\ 
scale~~~~~~~&~60~~~~~~~~~&s~~~~~~~~~~~&1~~~~~~~~~~~&4~~~~~~~~~~~&~5~~~~~~~~~~&i~~~~~~~~~~~\\ 
hole~~~~~~~~&~58~~~~~~~~~&h~~~~~~~~~~~&1~~~~~~~~~~~&3~~~~~~~~~~~&~4~~~~~~~~~~&i~~~~~~~~~~~\\ 
stepbrother~&~~0~~~~~~~~~&s~~~~~~~~~~~&3~~~~~~~~~~~&9~~~~~~~~~~~&11~~~~~~~~~~&a~~~~~~~~~~~\\ 
beggar~~~~~~&~~2~~~~~~~~~&b~~~~~~~~~~~&2~~~~~~~~~~~&4~~~~~~~~~~~&~6~~~~~~~~~~&a~~~~~~~~~~~\\ 
loin~~~~~~~~&~~1~~~~~~~~~&l~~~~~~~~~~~&1~~~~~~~~~~~&3~~~~~~~~~~~&~4~~~~~~~~~~&i~~~~~~~~~~~\\ 
survey~~~~~~&~37~~~~~~~~~&s~~~~~~~~~~~&2~~~~~~~~~~~&4~~~~~~~~~~~&~6~~~~~~~~~~&i~~~~~~~~~~~\\ 
party~~~~~~~&216~~~~~~~~~&p~~~~~~~~~~~&2~~~~~~~~~~~&5~~~~~~~~~~~&~5~~~~~~~~~~&i~~~~~~~~~~~\\ 
traveler~~~~&~~8~~~~~~~~~&tr~~~~~~~~~~&3~~~~~~~~~~~&6~~~~~~~~~~~&~8~~~~~~~~~~&a~~~~~~~~~~~\\ 
screen~~~~~~&~48~~~~~~~~~&s~~~~~~~~~~~&1~~~~~~~~~~~&5~~~~~~~~~~~&~6~~~~~~~~~~&i~~~~~~~~~~~\\ 
growth~~~~~~&155~~~~~~~~~&g~~~~~~~~~~~&1~~~~~~~~~~~&4~~~~~~~~~~~&~6~~~~~~~~~~&i~~~~~~~~~~~\\ 
cave~~~~~~~~&~~9~~~~~~~~~&k~~~~~~~~~~~&1~~~~~~~~~~~&3~~~~~~~~~~~&~4~~~~~~~~~~&i~~~~~~~~~~~\\ 
potato~~~~~~&~15~~~~~~~~~&p~~~~~~~~~~~&3~~~~~~~~~~~&6~~~~~~~~~~~&~6~~~~~~~~~~&i~~~~~~~~~~~\\ 
flour~~~~~~~&~~8~~~~~~~~~&f~~~~~~~~~~~&1~~~~~~~~~~~&4~~~~~~~~~~~&~5~~~~~~~~~~&i~~~~~~~~~~~\\ 
performance~&122~~~~~~~~~&p~~~~~~~~~~~&3~~~~~~~~~~~&9~~~~~~~~~~~&11~~~~~~~~~~&i~~~~~~~~~~~\\ 
manager~~~~~&~88~~~~~~~~~&m~~~~~~~~~~~&3~~~~~~~~~~~&6~~~~~~~~~~~&~7~~~~~~~~~~&a~~~~~~~~~~~\\ 
beauty~~~~~~&~71~~~~~~~~~&b~~~~~~~~~~~&2~~~~~~~~~~~&5~~~~~~~~~~~&~6~~~~~~~~~~&i~~~~~~~~~~~\\ 
elevator~~~~&~12~~~~~~~~~&E~~~~~~~~~~~&4~~~~~~~~~~~&7~~~~~~~~~~~&~8~~~~~~~~~~&i~~~~~~~~~~~\\ 
horse~~~~~~~&117~~~~~~~~~&h~~~~~~~~~~~&1~~~~~~~~~~~&4~~~~~~~~~~~&~5~~~~~~~~~~&a~~~~~~~~~~~\\ 
socks~~~~~~~&~~7~~~~~~~~~&s~~~~~~~~~~~&1~~~~~~~~~~~&4~~~~~~~~~~~&~5~~~~~~~~~~&i~~~~~~~~~~~\\ 
referee~~~~~&~~1~~~~~~~~~&r~~~~~~~~~~~&3~~~~~~~~~~~&5~~~~~~~~~~~&~7~~~~~~~~~~&a~~~~~~~~~~~\\ 
hair~~~~~~~~&148~~~~~~~~~&h~~~~~~~~~~~&1~~~~~~~~~~~&3~~~~~~~~~~~&~4~~~~~~~~~~&i~~~~~~~~~~~\\ 
boys~~~~~~~~&143~~~~~~~~~&b~~~~~~~~~~~&1~~~~~~~~~~~&3~~~~~~~~~~~&~4~~~~~~~~~~&a~~~~~~~~~~~\\ 
friend~~~~~~&133~~~~~~~~~&f~~~~~~~~~~~&1~~~~~~~~~~~&5~~~~~~~~~~~&~6~~~~~~~~~~&a~~~~~~~~~~~\\ 
foreigners~~&~13~~~~~~~~~&f~~~~~~~~~~~&3~~~~~~~~~~~&7~~~~~~~~~~~&10~~~~~~~~~~&a~~~~~~~~~~~\\ 
rangers~~~~~&~~2~~~~~~~~~&r~~~~~~~~~~~&2~~~~~~~~~~~&6~~~~~~~~~~~&~7~~~~~~~~~~&a~~~~~~~~~~~\\ 
wall~~~~~~~~&160~~~~~~~~~&w~~~~~~~~~~~&1~~~~~~~~~~~&3~~~~~~~~~~~&~4~~~~~~~~~~&i~~~~~~~~~~~\\ 
lamb~~~~~~~~&~~7~~~~~~~~~&l~~~~~~~~~~~&1~~~~~~~~~~~&3~~~~~~~~~~~&~4~~~~~~~~~~&a~~~~~~~~~~~\\ 
librarian~~~&~~5~~~~~~~~~&l~~~~~~~~~~~&3~~~~~~~~~~~&9~~~~~~~~~~~&~9~~~~~~~~~~&a~~~~~~~~~~~\\ 
blind~~~~~~~&~47~~~~~~~~~&b~~~~~~~~~~~&1~~~~~~~~~~~&5~~~~~~~~~~~&~5~~~~~~~~~~&a~~~~~~~~~~~\\ 
peaches~~~~~&~~1~~~~~~~~~&p~~~~~~~~~~~&2~~~~~~~~~~~&5~~~~~~~~~~~&~7~~~~~~~~~~&a~~~~~~~~~~~\\ 
plums~~~~~~~&~~1~~~~~~~~~&p~~~~~~~~~~~&1~~~~~~~~~~~&5~~~~~~~~~~~&~5~~~~~~~~~~&a~~~~~~~~~~~\\ 
guests~~~~~~&~62~~~~~~~~~&g~~~~~~~~~~~&1~~~~~~~~~~~&5~~~~~~~~~~~&~6~~~~~~~~~~&a~~~~~~~~~~~\\ 
building~~~~&160~~~~~~~~~&b~~~~~~~~~~~&2~~~~~~~~~~~&6~~~~~~~~~~~&~8~~~~~~~~~~&i~~~~~~~~~~~\\ 
puppy~~~~~~~&~~1~~~~~~~~~&p~~~~~~~~~~~&2~~~~~~~~~~~&4~~~~~~~~~~~&~5~~~~~~~~~~&a~~~~~~~~~~~\\ 
elves~~~~~~~&~~1~~~~~~~~~&E~~~~~~~~~~~&1~~~~~~~~~~~&4~~~~~~~~~~~&~5~~~~~~~~~~&a~~~~~~~~~~~\\ 
\hline
\caption{List of English noncognate control words}\label{eng_controls}
\end{longtable}

%% \end{centering}

\subsection{Out of context word naming}
A Spanish out of context word naming task was administered to second language learners of Spanish in order to assess the degree to which the target cognates were sensitive in eliciting parallel activation of English and Spanish (i.e., experiment in Chapter \ref{ESOOC}). At the beginning of each trial, a fixation point was displayed until the participant pressed a key. The fixation point was followed by a Spanish target word. Upon the display of each word, participants were told to name the  target into a voice trigger microphone as quickly and as accurately as possible. Their naming session was recorded to access naming accuracy following the task. Participants saw each of the 64 cognates and 64 controls for a total of 128 items. They also saw 12 practice items at the beginning of the experiment. The items were pseudo-randomized prior to each session with the constraint that the  participants should never see more than three cognates or noncognates in a row. 

%An English out of context experiment was also administered to native English monolinguals (Experiment in Chapter \ref{EOOC}). This was done in order to ensure that any cognate effects found in the Spanish out of context study could truly be associated to parallel activation of Spanish and English, and not to lexical properties of the stimuli. Because English monolinguals have no knowledge of Spanish (or any other language), naming latencies for cognates should not differ from those of noncognates. The procedure for the English out of context experiment was identical to that of the Spanish out of context experiment, except for the language of the stimuli. 

Word naming was chosen because prior studies show that it is a sufficiently sensitive task for detecting parallel activation of two languages \parencite[e.g.,][]{Schwartz2007}. Furthermore, overt naming, in comparison to a lexical decision task, ensures that participants activate the target word in the language of the task because they are required to speak in that language. In contrast, for a lexical decision task, one could argue that for any given cognate, a participant can respond ``yes'' upon identifying the cognate as a word in either language (especially if no false cognates are present to deter this behavior).

\subsection{In context: RSVP task}\label{Roadmap::WordNaming::InContext}
The in context Rapid Serial Visual Presentation (RSVP) task allowed for the assessment of parallel activation while participants read sentences. In this instantiation of the RSVP task, participants were presented with a fixation cross at the beginning of each trial. After the participant pressed a key, a sentence was displayed word-by word at a fixed pace. When the target word, marked in red, appeared it remained on the screen until the participant spoke the word into the voice trigger microphone. At this point, the remainder of the sentence was displayed, word-by-word. One quarter of the sentences contained yes--no comprehension sentences. RTs to name the target word and measures of accuracy for both naming and comprehension questions were recorded. Thus, the dependent measures for the RSVP task are the same as the measures in the out of context word naming task.

The Spanish target words (64 cognates and 64 controls), originally chosen for  the out of context task, were embedded into sentence contexts of high and low syntactic constraint (all sentences were low semantic constraint) yielding a total of 256 Spanish sentences. These Spanish sentences were translated into English in a manner such that all sentences were low syntactic constraint, though their specificity coding was preserved from the Spanish materials to provide a control comparison. The Spanish and English versions of the materials were counterbalanced into two versions with each language comprising a separate block. 

The Spanish version of the materials allowed for the comparison of cognate status, syntactic constraint, and the interaction between the two. The English version of the materials provided the comparison of cognate status. The syntactic constraint manipulation in the English translations served as a control for the Spanish constraint manipulation. Because all sentences were translated into English to be low syntactic constraint, no differences in the size of the cognate effect should be observed across constraint conditions, ensuring that any modulation due to syntactic constraint can really be attributed to the syntax and not to extraneous properties of the words or sentences.

There were three versions of the RSVP experiment in Chapter \ref{SEInContext} intended for three groups of bilinguals: Spanish-English bilinguals, English monolinguals, and Spanish monolinguals. The bilinguals read sentences in both languages across two separate blocks. The Spanish and English monolingual speakers read sentences in Spanish and English respectively. The justification for including each group of participants is provided above in Section \ref{Roadmap::Participants}.

The materials were counterbalanced in such a way that a single participant never saw the translation of any sentence (or the same sentence) across blocks. For example, if a bilingual participant saw a cognate and its matched control in the low syntactic constraint condition in English, when they read the Spanish sentences the target and control would appear in a high constraint sentence. They would receive the opposite conditions with a different pair of target words. This method of counterbalancing does mean that  each participant saw every cognate and control twice across blocks (once in each language for bilinguals; twice in the same language for monolinguals). 48 filler sentences were added to each language block. The blocks were pseudorandomized before each session such that no condition (cognate high constraint, cognate low constraint, etc.) was ever repeated more than three times in a row.   

The RSVP task has been used successfully to demonstrate evidence for parallel activation in sentence context  \parencite[e.g.,][]{Schwartz2006}. While it is less naturalistic than the eye-tracking methodology, it accurately taps into the word-recognition process while at the same time providing a less complex dataset for analysis. Also, previous studies show that RSVP can yield  results similar to eye-tracking \parencite[][]{Altarriba1996}. Furthermore, the dependent  measure for RSVP is the same as the one used in the out of context norming experiment (i.e., time to begin naming the target), allowing for comparison between the in context and out of context results.

\section{Operation Span task}
The Operation Span task \parencite[][]{Turner1989} was included as a measure of working memory. In this task, participants judge whether sets of equations, presented one at a time, were correct or incorrect (e.g., 5 X 2 + 1 = 10). After each equation, they were given a word to memorize. At the end of a set, they were prompted to recall as many words as possible from the set. Participants saw a total of 15 sets of equations and words increasing in difficulty as they proceeded through the experiment. A participant’s “operation span” was calculated by the number of words correctly recalled for trials on which their performance on the numerical problem was correct. 

Because participants in the set of experiments differed in their language experience, participants were allowed to perform the task in whichever language they felt was most comfortable or dominant. Thus, there were two versions of the Simon task: English and Spanish. Each version contained the same equations, and the words were translated across languages. 

The justification for including the Operation Span task was to balance the groups of participants in terms of their working memory. It can also be used to investigate the way in which working memory span influences processing, given that some of our manipulations may depend on working memory (e.g., ability to maintain a syntactic representation in memory over the course of a long sentence).  

\section{Simon task}
The Simon task \parencite[][]{Simon1967} was included as a measure of inhibitory control. In this task,  participants must inhibit one form of information (spatial location) in favor of another (color). They saw a series of boxes colored either red or blue. The boxes appeared in the left-, middle-, or right-hand side of the screen. There were a total of 126 trials across three blocks. In each block, participants saw seven iterations of each of the six individual conditions (red or blue by left, middle, or right) in a random order. They were instructed  to quickly press a colored button (blue on the left, red on the right) corresponding to the color of the box on the screen. On the critical trials the required key press (left- or right-hand side) conflicted with the side that the box was on.  By comparing the reaction times to respond to the conflicting cues from the reaction time to respond to congruent cues, one can obtain a measure of the ease to which participants could inhibit the irrelevant cue (position) and attend to the relevant cue (color). 

The primary reason for including the Simon task was to control cognitive function across groups of speakers. However, previous studies have shown that bilingual groups can outperform monolingual groups on tasks of executive function.  

\section{Picture naming task}
Participants performed a picture naming task as a means of assessing English proficiency. Participants saw a total of 72 line drawings, presented one at a time, that they were instructed to name aloud. The time to begin naming the picture as well as the accuracy of naming were used as the dependent measures. The pictures were drawn from a variety of semantic categories (tools, instruments, clothing, furniture, etc.)  One third of the pictures had cognate names in English and Spanish. 

The naming accuracy and reaction time to begin naming should reflect the proficiency a group of speakers has with English. More proficient English speakers should, on average, name faster and more accurately compared to less proficient speakers. Additionally, more proficient speakers of English may demonstrate less of a cognate effect during English naming. In this manner, the English picture naming task can be used to compare naming performance of native speakers with the performance of L2 speakers. 

\section{Grammar tests}
In order to assess language proficiency in both English and Spanish, bilinguals performed portions of two grammar tests: the Michigan English Language Institute College English Test (MELICET) and the Diplomas de Espa\~{n}ol como Lengua Extranjera (DELE). Each portion  contained 50 questions. Each test covered grammatical aspects such as verb conjugation and preposition choice. All questions were multiple choice. 

While the grammar tests will not provide a comparison of the relative proficiency of each language, they can be used to compare groups of speakers within languages, in a similar  manner as the picture naming task will be used. Thus, more proficient speakers of either language should score more highly, on average, compared to speakers who are less proficient in that language.

\section{Predictions}
To review, there are four main experimental sessions for the current investigation: (1) out of context word naming with Spanish-English bilinguals (Chapter \ref{ESOOC}); (2) in context naming with Spanish-English bilinguals (Chapter \ref{SEInContext}); (3) in context naming with English monolinguals (Chapter \ref{SEInContext}); and (4) in context naming with Spanish monolinguals (Chapter \ref{SEInContext}).  The predictions for each experiment are reviewed in the following paragraphs. 

In order to provide a sense of the degree to which the target cognates that were chosen for the present investigation are able to activate two languages, native English speakers who are L2 learners of Spanish will name words in Spanish (Chapter \ref{ESOOC}).  If the cognates are sufficiently able to activate both Spanish and English, then the group of Spanish learners  should produce a reliable cognate effect. That is, cognates should be named significantly faster compared to the matched noncognate controls. 

The presence of the cognate effect would replicate a finding from a long line of previous research that suggests that bilinguals activate both of their language in parallel while reading words in one language. If the cognate effect does not emerge, it would suggest that some property of the stimuli was not sufficiently controlled and interfered with the he measurement of parallel activation.

To investigate bilingual word recognition in context, native Spanish speakers who have learned English as a second language will participate in the main experiment of the investigation (Chapter \ref{SEInContext}: RSVP sentence reading with word naming). To review, in this task each participant will see two language blocks (English and Spanish). Within each language block they will name cognates and noncognates embedded in syntax nonspecific or syntax specific sentences (though the syntax specific condition in English is a control manipulation and is actually language-general).  All sentences contain low semantic constraints to avoid confounding the possible effects of the syntax with effects of semantic bias. Previous studies find cognate effects in low constraint sentences, suggesting that bilinguals activate both languages while reading unilingual sentences. Therefore, in our sentences that contain language-general syntax, the same results are predicted. A facilitatory cognate effect is expected in syntax nonspecific sentences of Spanish and in all English sentences (due to the dummy manipulation). 

The predictions in the language-specific conditions are more open-ended. If language-specific syntax can cue bilinguals into the language of the sentence and allow them to access the meaning of cognates selectively, then the cognate effect should be reduced or eliminated following the syntactic constraint manipulation compared to sentences with language-general syntax. In this case, an interaction between cognate status and syntactic specificity is predicted. In contrast, if there is a limited role of context in influencing word recognition, there should be no difference in the size of the cognate effect between language-specific and language-general syntax conditions. In this case, only a main effect of cognate status for Spanish sentences is predicted. 

In order to ensure that any effects found in the in context task are due to the intended manipulations, two control RSVP experiments will be conducted with monolingual speakers of English and Spanish.  Because English monolinguals have no knowledge of Spanish to activate in parallel, no cognate effect is predicted. Furthermore, to the extent that there are no differences that affect word recognition between the language-specific and language-nonspecific sentences, there should be no main effect of the English syntax dummy  manipulation. Spanish monolinguals will read the Spanish sentences from the main RSVP experiment. Similarly to the English monolinguals, the Spanish monolinguals have no knowledge of English and should thus exhibit no cognate effect of while naming Spanish words in context. If there are syntactic differences between the Spanish-specific syntax and the language-general syntactic conditions that influence word recognition, they should emerge for the Spanish monolinguals as well as for the Spanish-English bilinguals. 
