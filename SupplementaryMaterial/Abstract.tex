% Place abstract below.
Many recent studies demonstrate that bilingual word recognition is language nonselective in nature. Bilinguals activate information about words in each language in parallel when reading or listening to a word in one language alone \parencite[e.g.,][]{Dijkstra2005,Marian2003}, even if the word is embedded in a sentence context \parencite[e.g.,][]{Duyck2007, Libben2009, Schwartz2006,VanHell2008}. Two factors have been identified that effectively eliminate the cross-language effect within sentence context: a highly biased semantic constraint \parencite[e.g.,][]{Schwartz2006} and when words differ in grammatical class across both languages \parencite[e.g.,][]{Baten2010}.

One contextual factor that has been ignored is the presence of language-specific syntactic constraints. In the current study, highly proficient Spanish-English bilinguals read sentences in each language across separate blocks. Half of the Spanish sentences contained syntax that was structurally specific to Spanish in two ways: (a) the indirect object of a ditransitive verb was realized pleonastically with the proclitic le and its corresponding noun phrase, and (b) the grammatical subject of the object relative clause was not expressed overtly (e.g., Las monjas (a)le llevaron las mantas que (b)(pro) hab\'{i}an bordado a la directora del orfanato. [The nuns took the quilts that they had embroidered to the director of the orphanage.]) The English translations were controls in that the initial phrase of the sentence was not syntactically specific to either language. Bilinguals read sentences presented word-by-word and named a target word aloud. Critical target words were language-ambiguous cognates (e.g., bus in English and Spanish), which were matched to a set of unambiguous control words (e.g., hairspray-laca).

The results indicated that language-specific syntactic constraints did not reliably modulate the cognate effect for all bilinguals. This suggests that the bilinguals activated both languages even in sentences with syntactic structure specific to only one language. However, a subset of the bilinguals, those who were dominant in Spanish, did appear to make use of the syntactic constraints to switch off the unintended language. However, there was not sufficient power to find an effect statistically. The current study shows that it may be premature to conclude that language-specific syntactic constraints do not modulate nonselectivity. The implications for models of the bilingual lexicon are discussed.
