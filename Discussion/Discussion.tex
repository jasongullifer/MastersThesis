\chapter{General Discussion}
The current study sought to examine bilingual word recognition in sentence context. The first goal was replicate previous studies that find support for nonselectivity in unilingual sentences. The second goal was to investigate whether syntactic constructions that are specific to only one of a bilingual's two language can reduce or eliminate the parallel activation. To accomplish these goals, two experiments were designed: an out of context naming experiment (Chapter \ref{ESOOC}) and an RSVP naming experiment in sentence context (Chapter \ref{SEInContext}). Overall, the results showed that bilinguals, but not monolinguals, activated both languages in parallel, as evidenced by the cognate effects, even in the presence of a sentence context.  Parallel activation persisted even in  sentences that  were syntactically specific to one language. Interestingly, there was preliminary evidence that a sub-group of the participants appeared to exploit the presence of language-specific syntactic constraints to modulate nonselectivity. For the other participants, parallel activation persisted in the specific sentences. Because they are highly speculative, the results from these two groups are presented in this section  All results will be interpreted under the framework of the BIA+ model \parencite[][]{Dijkstra2002}.

In Chapter \ref{ESOOC}, L2 learners of Spanish showed a facilitatory cognate effect for words named in Spanish. This result was interpreted as evidence for parallel activation of two languages. If the bilinguals were activating only Spanish without considering words in English, then the bilinguals should not have been facilitated due to the convergence of orthography,  phonology, and semantics of the cognates across languages. Overall, the first experiment confirmed that the target words were sensitive to the detection of parallel activation. Hence, it was appropriate to use these words to investigate whether word recognition is also nonselective in L1 word recognition in sentence context. 

In Chapter \ref{SEInContext}, native Spanish speakers who were proficient in English showed a facilitatory cognate effect in both languages. Neither  Spanish monolingual speakers nor English monolinguals displayed cognate effects while naming words embedded in sentences. This shows that the naming  latencies in English and in Spanish sentences were not influenced by unintended  lexical properties of the targets or sentential properties of the sentences. As of yet, no other study examining bilingual word recognition in sentence context has included a monolingual control in both languages. Thus, in the present investigation, we can be relatively certain that the cognate effects that occurred in sentence context (and in isolation) are due to parallel activation of two languages.

This is also the first investigation into  the role of language-specific syntax for bilingual word recognition. In the current experiment, there was no evidence that language-specific syntax modulated magnitude of the cognate effect. If a sentence is constructed using syntactic constructions available only in one language, words in both languages are still activated. The finding that syntax does not influence parallel activation is compatible with BIA+ model \parencite[][]{Dijkstra2002}. Word recognition in the BIA+ model is a bottom-up, data driven process. Prior expectations about  the language of a sentence do not influence word recognition in the BIA+ model. If  language-specific syntax in the present study were able to enhance  expectations about the language of the task, then it did not come into play inside the word recognition system. Otherwise, nonselectivity should have been influenced by the specific syntax.  Instead, any expectations established by the language-specific syntax  came into play likely at the level of the task schema. The task schema is modularized from the word recognition system, and it is not allowed to impede on word recognition.  

Yet, previous studies have found that parallel activation can be overcome by a highly biased semantic constraint or when words differ in their lexical properties across languages \parencite[e.g.,][]{Baten2010,Schwartz2006,Sunderman2006}. So it is possible for bilinguals to switch off the unintended language. There are at least three possible reasons why the bilinguals may not have shown a modulation of the cognate effect in the sentences with specific syntax. First, language-specific syntax may not function in the predicted manner of constraining nonselectivity. One alternative is that it does function in the predicted manner, but that the bilinguals were not influenced by the function of the syntax, either because the structures chosen were not powerful enough, or because the bilinguals  were not sensitive. It is possible that because the bilinguals were all immersed in an English speaking environment, they became less sensitive to the specific syntax. Previous work demonstrates that bilinguals who are immersed in an L2 environment can experience changes in their parsing preferences \parencite[e.g.,][]{Dussias2003,Dussias2007}. Yet another alternative is that the bilinguals are sensitive to language-specific syntactic constraints, but that the sentences were too long for participants to maintain the constraints in memory until they reached the target word. However, given that the monolinguals showed the main effect of syntactic specificity, the appearance of this effect is likely not due to working memory difference. An analysis of working memory scores and appearance of the specificity effect yielded no significant correlations.

In the current data set, there is a clue to this issue of immersion shifting syntactic processing. The Spanish monolinguals, who are all immersed in an L1 environment, process the specific syntax conditions faster, yet the bilingual speakers do not show this facilitation as a group. In order to investigate the alternative that the bilinguals are not sensitive to the language-specific syntax, the data were reexamined. Some bilinguals processed the sentences with specific syntax like monolingual speakers, and some did not.  Furthermore, those who did process like native speakers, tended to be Spanish dominant. When the data  were analyzed by these two groups, the bilinguals who mirrored the Spanish monolinguals began to show our predicted interaction, though the sample size was too small to find significant effects. 

%%%%%%%%%%%%%%%%%%%%%%%%%%%%%
%The lack of a main effect of specificity indicated that the manipulation may not have been utilized by the bilingual participants. Yet,  the monolingual Spanish speakers did show a hint of a facilitation in naming latencies in the specific and nonspecific sentences. To investigate  processing under the sentences containing specific syntax, the bilinguals were split into two groups, those who, like the Spanish monolinguals, became facilitated in the syntax specific condition and those who did not.  This division effectively separated the speakers who may have been processing the specific sentences like native speakers from those who were not. 

Naming latencies for the bilinguals who processed the specific condition like native speakers are shown in Table \ref{spec_faster_means}. The naming latencies for the bilinguals who did not process like native speakers are shown in Table \ref{spec_slower_means}. Post-hoc repeated measures ANOVAs were  conducted on each of these groups. Participants who were faster in the specific condition showed a marginal main effect of language (\textit{F}(1,7) $=$ 4.50, \textit{p} $=$ 0.072) with Spanish words being named faster than English words (\textit{M}$_{Spanish} =$ 727 ms, \textit{M}$_{English} =$ 764 ms). They also showed a significant main effect of syntactic specificity (\textit{F}(1,7) $=$ 10.52, \textit{p} $<$ 0.05); words in specific conditions were named faster than words in nonspecific conditions (\textit{M}$_{specific} =$ 734 ms; \textit{M}$_{nonspecific} =$ 756 ms). There was a marginal main effect of cognate status (\textit{M}$_{cognate} =$ 736 ms \textit{M}$_{noncognate} =$ 756 ms; \textit{F}(1,7) $=$ 5.53, \textit{p} $=$ 0.051). No two-way interactions were significant (language X specificity: \textit{F}(1,7) $=$ 1.94, \textit{p} $=$ 0.20; language X cognate status: \textit{F}(1,7) $=$ 2.49, \textit{p} $=$ 0.159; specificity X cognate status: \textit{F}(1,7) $<$ 1). The three way interaction between language, specificity, and cognate status approached significance (\textit{F}(1,7) $=$ 3.41, \textit{p} $=$ 0.107). 

\begin{table}[hptb]
\begin{center}
\begin{tabular}{|c|c|c|c|} \hline
\multicolumn{1}{|c|}{Condition}&\multicolumn{1}{c|}{Mean RT (ms)}&\multicolumn{1}{c|}{Std. Deviation}&\multicolumn{1}{c|}{N}\\ \hline\hline
English~Nonspecific~Cognate~~~~&767.36~~~&203.977~~~&8~\\ 
English~Nonspecific~Noncognate~&778.22~~~&186.238~~~&8~\\ 
\hline
English~Specific~Cognate~~~~~~~&734.07~~~&167.588~~~&8~\\ 
English~Specific~Noncognate~~~~&777.03~~~&220.379~~~&8~\\ 
\hline\hline
Spanish~Nonspecific~Cognate~~~~&730.68~~~&207.450~~~&8~\\ 
Spanish~Nonspecific~Noncognate~&750.45~~~&210.416~~~&8~\\ 
\hline
Spanish~Specific~Cognate~~~~~~~&709.96~~~&184.041~~~&8~\\ 
Spanish~Specific~Noncognate~~~~&717.57~~~&205.934~~~&8~\\ 
\hline
\end{tabular}
\caption{Mean naming latencies (in ms) in context for Spanish-English bilingual participants who processed Spanish-specific syntax like Spanish monolinguals}\label{spec_faster_means}
\end{center}
\end{table}


\begin{table}[hptb]
\begin{center}
\begin{tabular}{|c|c|c|c|} \hline
\multicolumn{1}{|c|}{Condition}&\multicolumn{1}{c|}{Mean RT (in ms)}&\multicolumn{1}{c|}{Std..Deviation}&\multicolumn{1}{c|}{N}\\ \hline\hline
English~Nonspecific~Cognate~~~~&634.39~~~&129.492~~~&7~\\ 
English~Nonspecific~Noncognate~&655.25~~~&150.049~~~&7~\\
\hline
English~Specific~Cognate~~~~~~~&638.80~~~&140.373~~~&7~\\
English~Specific~Noncognate~~~~&674.28~~~&164.431~~~&7~\\
\hline\hline
Spanish~Nonspecific~Cognate~~~~&647.51~~~&155.107~~~&7~\\
Spanish~Nonspecific~Noncognate~&667.85~~~&151.437~~~&7~\\
\hline
Spanish~Specific~Cognate~~~~~~~&666.18~~~&153.419~~~&7~\\
Spanish~Specific~Noncognate~~~~&708.48~~~&187.239~~~&7~\\
\hline
\end{tabular}
\caption{Mean naming latencies (in ms) in context for Spanish-English bilingual participants who did not process Spanish-specific syntax like Spanish monolinguals}\label{spec_slower_means}
\end{center}
\end{table}



Follow-up tests were not conducted because the effects were not significant, but the pattern of the means seems to indicate that in the Spanish sentences, the bilinguals who named words faster in specific condition show a cognate effect in the non-specific condition, but the cognate effect is eliminated in the specific condition. For English, it appears as though there is a cognate effect across the board, and that the effect is larger in the specific sentences.  

Participants who were not faster in the non-specific condition showed a significant main effect of specificity and of cognate status (specificity: \textit{F}(1,6) $=$ 12.90, \textit{p} $<$ 0.05; cognate status: \textit{F} $=$ 20.23, \textit{p} $<$ 0.01). Specific sentences were named slower than nonspecific sentences (\textit{M}$_{specific} =$ 671 ms; \textit{M}$_{nonspecific} =$ 651 ms) and cognates were named faster than noncognates (\textit{M}$_{cognate} =$ 647 ms; \textit{M}$_{noncognate} =$ 676 ms). The was no main effect of language (\textit{F}(1,6) $<$ 1). There was a significant interaction between specificity and cognate status. No other interactions were significant (language X specificity: \textit{F}(1,6) $=$ 1.16, \textit{p} $>$ 0.05; language X cognate status: \textit{F}(1,6) $<$ 1; language X specificity X cognate status: \textit{F}(1,6) $<$ 1). 

To explore the interaction between specificity and cognate status for the bilinguals who were slower in the specific sentences, simple effects tests were performed on the data collapsed over language. The simple effects tests revealed that in the nonspecific condition, cognates were named faster than noncognates (\textit{M}$_{cognate} =$ 641ms; \textit{M}$_{noncognate} =$ 661 ms; \textit{F}(1,6) $=$ 9.37, \textit{p} $<$ 0.05). In the specific condition, cognates were also named faster than noncognates, but the magnitude of the  effect was larger (\textit{M}$_{cognate} =$ 652 ms; \textit{M}$_{noncognate} =$ 691 ms; \textit{F}(1,6) $=$ 21.98, \textit{p} $<$ 0.01). Overall, it would appear that the manner in which bilinguals recognize words in the specific syntax condition influences whether the specific condition will modulate the cognate effect.
%%%%%%%%%%%%%%%%%%%%%%%%%%%%

%There were hints in the data that language-specific syntactic constraints can function to modulate parallel activation, but only for certain speakers.  This evidence is only speculative because the analyses lacked sufficient power to detect the effects. The Spanish monolinguals exhibited an effect of sentence type (specific or nonspecific) in that they were faster in the language-specific sentences. This facilitation is an indicator that something about the sentence type caused them to process the sentences with specific syntax differently than in the nonspecific condition. Therefore, there may be some information encoded in the language-specific sentences that the bilinguals could  exploit. 

%To explore this hypothesis, the bilinguals were then broken into two groups: those who processed  the Spanish specific syntax like native speakers and those who did not. The speakers who did not process specific sentences like the monolingual Spanish speakers showed a cognate effect across all conditions in both languages. These speakers did not utilize syntactic constraints during word recognition. The speakers who processed the specific condition like the Spanish monolinguals also showed a cognate effect in Spanish non-specific sentences. However, this effect appeared to be eliminated in the Spanish specific sentences. In the English sentences, the bilinguals  Given the small sample size after the bilingual group was broken down, there was not enough statistical power to find significant effects. The effects will become significant with a larger sample size.  If these results were to hold after the addition of more participants, they  have implications for the BIA+ model.

Currently, the BIA+ model has no explicit way to model syntactic effects on bilingual word recognition, though \textcite{Dijkstra2002} assert that syntactic effects may be able to influence word recognition in BIA+. In order to model the role of syntactic constraints, the nature of the constructions used in the current investigation must be explored. Experiment 2 included two types of syntactic constraint, the pro-clitic and pro-drop, and each one may function differently. The pro-clitic is a unique piece of the syntactic structure of Spanish and is  closely linked to the verb and its object. It may provide the reader with predictions about upcoming material in the sentence. In other words, when a reader encounters a proclitic, they expect to encounter a verb that takes an indirect object. While pro-drop is also unique to Spanish, it does not allow the reader to make predictions about the number of objects that a verb takes. For example, both intransitive and transitive verbs can  drop their subjects. Therefore, the presence of pro-drop may only provide the reader with additional expectations as to the language of the sentence. If the BIA+ model is correct in assuming that prior language expectations do not influence word recognition, then it would predict that only the pro-clitic should be able to reduce parallel activity of the unintended language.  If syntactic constraints ultimately turn out to be useful for bilinguals during word recognition, future research can explore this question by teasing apart the independent contributions of proclitics and pro-drop. 

One way for  BIA+ to model the effect of the proclitic would be to assume that syntactic information is encoded lexically \parencite[e.g.,][]{Pickering1998}. If syntactic information is present in the lexicon, the appearance of a proclitic could preactivate any verbs that use a proclitic as well as potential objects of the verb. This not only generates the prediction that both the verb and object would be processed more quickly, but because clitics exist in Spanish but not English, Spanish words would become more highly activated, allowing the reader to selectively access Spanish words before English can be activated. 

Another open question for the current investigation is what factors allow a bilingual to attend to the language-specific syntax. While no correlations were found for individual difference and language history  measures, an informal analysis of the data suggests that the bilinguals who were processing the specific sentence like native speakers tended to be Spanish dominant. Those who were English dominant tended not process like the Spanish monolinguals. This cross-tabulation is interesting in  light of research demonstrating that the parsing preferences of the native language are malleable  when bilinguals are immersed in an L2 environment \parencite[][]{Dussias2003,Dussias2007}. Because the syntactic constructions used in this experiment were specific to Spanish, the bilinguals who have been living in an English dominant environment experience less exposure to the constructions compared to bilinguals, or monolinguals, who are not immersed. Hence, they may be less able to exploit the Spanish-specific syntax to switch off the unintended language. Data are currently being collected on speakers who have more use of Spanish on a daily basis in order to more rigorously test this hypothesis. 

In conclusion, the results from the current experiments suggest that it may be premature to conclude that language-specific syntactic constraints have no influence on bilingual word recognition. More data need to be collected before a decision is made either way. 
